%-----------------------------------------------------------------------------------------------------------------------------------------------------------------------------------------------
% Conclusões: "fecha" com os objetivos? (respondem aos objetivos?) - aqui é que "se vende o peixe" - elas é que valorizam (ou não) o trabalho realizado.
% Normalmente é uma parte do trabalho "um pouco desprezada", pois o autor já está "cansado....". Mas é aqui que realmente se mede se o trabalho tem ou não valor.
% Contém o item Trabalhos futuros, que é uma orientação sobre as possibilidades de continuação do desenvolvimento do trabalho.
\chapter{Conclusão}
\label{Conclusao}

Acredita-se que o trabalho desenvolvido até então está em dia com o cronograma, visto que a interface gráfica desenvolvida encontra-se totalmente funcional e o algoritmo detecta e segmenta relativamente bem os obstáculos dos cenários propostos. Entretanto, a taxa de processamento tornou-se uma preocupação, visto que o algoritmo desenvolvido consegue processar a uma taxa de captura de aproximadamente oito quadros por segundo. A alteração de plataforma poderia ser uma solução para o aumento da taxa de atualização, visto que o código implementado nos sistema embarcados dispensam interface gráfica e diversas outras funcionalidades, como exemplo a funcionalidade para reconstrução tridimensional do ambiente. Todavia, essas plataformas não são tão poderosas quanto o computador da estação-base, logo, possivelmente essas plataformas apresentem performance inferior.

Deste modo, duas providencias  são cruciais para a continuidade do trabalho. A primeira é a otimização das rotinas utilizadas pelo algoritmo, sendo essa uma tentativa para a melhora do desempenho geral do método. A segunda é a execução de uma ampla revisão bibliográfica, incluindo principalmente trabalhos que apresentem comparativos de desempenho entre plataformas e que tenham realizados algum tipo de aceleração por \textit{hardware}, sejam eles envolvendo paralelização de processos, implementação em FPGA ou utilizando a plataforma de computação paralela CUDA (\textit{Compute Unified Device Architecture}). Deste modo, será possível identificar a plataforma mais adequada para este propósito. 

Atividades Futuras:

\begin{itemize}
	\item Otimizar Código
	\item Alterar Interface gráfica para suportar o Método SGBM
	\item Implementar Segmentação por Distância/Limiarização do mapa de disparidades utilizando Método de Otsu
	\item Aprimorar a suavização da Nuvem de Pontos Gerada
	\item Analisar Métodos de Abertura e Fechamento para aprimoramento na Segmentação
	\item Implementar funções para medir a Distância de objetos próximos
	\item Reavaliar a Calibração das Câmeras	
	\item Realizar ensaio de Ground Truth para validação da Distância Detectada
	\item Implementar os algoritmos de Visão na BeagleBone e Jetson TK1
	\item Implementar os algoritmos de Visão na Jetson TK1
	\item Comparação de Desempenho: Desktop x BBB x Jetson TK1
\end{itemize}








