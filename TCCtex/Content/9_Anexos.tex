%------------------------------------- Anexo 1 -------------------------------------------------------
\chapter{Anexo 1}
\label{Anexo1}

\textcolor{red}{Dar Créditos à pela parte 3D Reconstruction: http://opencv.jp/opencv2-x-samples/point-cloud-rendering}

\textcolor{red}{Dar Créditos à Kyle Hounslow pela parte do Tracking Object}

\lstset {language=C++}
\textbf{trackObject.h}
\begin{lstlisting}
/*
 * trackObject.h
 *
 *  Author: Kyle Hounslow
 * 	Link: https://www.youtube.com/watch?v=bSeFrPrqZ2A
 *  Published in: March 11, 2013
 */

#ifndef SRC_TRACKOBJECT_H_
#define SRC_TRACKOBJECT_H_

/* Libraries */
#include "opencv2/opencv.hpp"

using namespace cv;
using namespace std;

//default capture width and height
const int FRAME_WIDTH = 640;
const int FRAME_HEIGHT = 480;
//max number of objects to be detected in frame
const int MAX_NUM_OBJECTS=50;
//minimum and maximum object area
const int MIN_OBJECT_AREA = 20*20;
const int MAX_OBJECT_AREA = FRAME_HEIGHT*FRAME_WIDTH/1.5;

string intToString(int number);
void drawObject(int x, int y,Mat &frame);
void trackFilteredObject(int &x, int &y, Mat threshold, Mat &cameraFeed);

#endif /* SRC_TRACKOBJECT_H_ */
\end{lstlisting}

\textbf{trackObject.cpp}
\begin{lstlisting}
/*
 * trackObject.cpp
 *
 *  Created on: Oct 19, 2015
 *      Author: nicolasrosa
 */

#include "trackObject.h"

void trackFilteredObject(int &x, int &y, Mat threshold, Mat &cameraFeed){

	Mat temp;
	threshold.copyTo(temp);
	//these two vectors needed for output of findContours
	std::vector< std::vector<Point> > contours;
	std::vector<Vec4i> hierarchy;
	//find contours of filtered image using openCV findContours function
	findContours(temp,contours,hierarchy,CV_RETR_CCOMP,CV_CHAIN_APPROX_SIMPLE );
	//use moments method to find our filtered object
	double refArea = 0;
	bool objectFound = false;
	if (hierarchy.size() > 0) {
		int numObjects = hierarchy.size();
        //if number of objects greater than MAX_NUM_OBJECTS we have a noisy filter
        if(numObjects<MAX_NUM_OBJECTS){
			for (int index = 0; index >= 0; index = hierarchy[index][0]) {

				Moments moment = moments((cv::Mat)contours[index]);
				double area = moment.m00;

				//if the area is less than 20 px by 20px then it is probably just noise
				//if the area is the same as the 3/2 of the image size, probably just a bad filter
				//we only want the object with the largest area so we safe a reference area each
				//iteration and compare it to the area in the next iteration.
                if(area>MIN_OBJECT_AREA && area<MAX_OBJECT_AREA && area>refArea){
					x = moment.m10/area;
					y = moment.m01/area;
					objectFound = true;
					refArea = area;
				}else objectFound = false;


			}
			//let user know you found an object
			if(objectFound ==true){
				putText(cameraFeed,"Tracking Object",Point(0,50),2,1,Scalar(0,255,0),2);
				//draw object location on screen
				drawObject(x,y,cameraFeed);}

		}else putText(cameraFeed,"TOO MUCH NOISE! ADJUST FILTER",Point(0,50),1,2,Scalar(0,0,255),2);
	}
}

void drawObject(int x, int y,Mat &frame){

	//use some of the openCV drawing functions to draw crosshairs
	//on your tracked image!

    //UPDATE:JUNE 18TH, 2013
    //added 'if' and 'else' statements to prevent
    //memory errors from writing off the screen (ie. (-25,-25) is not within the window!)

	circle(frame,Point(x,y),20,Scalar(0,255,0),2);
    if(y-25>0)
    line(frame,Point(x,y),Point(x,y-25),Scalar(0,255,0),2);
    else line(frame,Point(x,y),Point(x,0),Scalar(0,255,0),2);
    if(y+25<FRAME_HEIGHT)
    line(frame,Point(x,y),Point(x,y+25),Scalar(0,255,0),2);
    else line(frame,Point(x,y),Point(x,FRAME_HEIGHT),Scalar(0,255,0),2);
    if(x-25>0)
    line(frame,Point(x,y),Point(x-25,y),Scalar(0,255,0),2);
    else line(frame,Point(x,y),Point(0,y),Scalar(0,255,0),2);
    if(x+25<FRAME_WIDTH)
    line(frame,Point(x,y),Point(x+25,y),Scalar(0,255,0),2);
    else line(frame,Point(x,y),Point(FRAME_WIDTH,y),Scalar(0,255,0),2);

	putText(frame,intToString(x)+","+intToString(y),Point(x,y+30),1,1,Scalar(0,255,0),2);

}

string intToString(int number){
    stringstream ss;
	ss << number;
	return ss.str();
}
\end{lstlisting}

%------------------------------------- Anexo 2 -------------------------------------------------------
\chapter{Anexo 2}
\label{Anexo2}

Texto do Anexo 2.



