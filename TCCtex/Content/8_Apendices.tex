%------------------------------------- Ap�ndice 1  --------------------------------------------------
\chapter{Ap�ndice 1}
\label{Apendice1}

%%Remover
%\textcolor{red}{\textbf{Dica:} Diferen�a entre Ap�ndice e Anexo:
%	\begin{itemize}
%		\item AP�NDICE -- Documento ou texto elaborado pelo autor
%		\item ANEXO -- Documento ou texto n�o elaborado pelo autor
%	\end{itemize}
%}

Abaixo est�o apresentados todos os trechos de c�digo desenvolvido para este trabalho. Essa se��o est� subdividada organizada em duas partes, bibliotecas e c�digo-fonte, para cada uma das plataformas utilizadas.

\section{Interface Gr�fica - StereoVision System}
\subsection{Bibliotecas}

\lstset{language=C++}
\textbf{3DReconstruction.h}
\begin{lstlisting}
/*
 * 3DReconstruction.h
 *
 *  Created on: Oct 25, 2015
 *      Author: nicolasrosa
 */

#ifndef RECONSTRUCTION_3D_H
#define RECONSTRUCTION_3D_H

/* Libraries */
#include "opencv2/opencv.hpp"

using namespace cv;
using namespace std;

/* 3D Reconstruction Classes */
template <class T>
static void projectImagefromXYZ_(Mat& image, Mat& destimage, Mat& disp, Mat& destdisp, Mat& xyz, Mat& R, Mat& t, Mat& K, Mat& dist, Mat& mask, bool isSub);

template <class T>
static void fillOcclusionInv_(Mat& src, T invalidvalue);

template <class T>
static void fillOcclusion_(Mat& src, T invalidvalue);

// 3D Reconstruction
class Reconstruction3D{
public:
    Reconstruction3D(); //Constructor
    void setViewPoint(double x,double y,double z);
    void setLookAtPoint(double x,double y,double z);
    void PointCloudInit(double baseline,bool isSub);

    /* 3D Reconstruction Functions */
    void eular2rot(double yaw,double pitch, double roll,Mat& dest);
    void lookat(Point3d from, Point3d to, Mat& destR);
    void projectImagefromXYZ(Mat &image, Mat &destimage, Mat &disp, Mat &destdisp, Mat &xyz, Mat &R, Mat &t, Mat &K, Mat &dist, bool isSub);
    void fillOcclusion(Mat& src, int invalidvalue, bool isInv);


    Mat disp3Dviewer;
    Mat disp3D;
    Mat disp3D_8U;
    Mat disp3D_BGR;

    Point3d viewpoint;
    Point3d lookatpoint;


    Mat dist;
    Mat Rotation;
    Mat t;

    Mat depth;

    double step;
    bool isSub;
};


#endif // RECONSTRUCTION_3D_H
\end{lstlisting}

\textbf{mainwindow.h}
\begin{lstlisting}
/*
 * mainwindow.h
 *
 *  Created on: Oct 1, 2015
 *      Author: nicolasrosa
 */

#ifndef MAINWINDOW_H
#define MAINWINDOW_H

/* Libraries */
#include <QMainWindow>
#include <opencv2/opencv.hpp>

/* Custom Libraries */
#include "StereoProcessor.h"
#include "setstereoparams.h"

using namespace cv;

namespace Ui{
    class MainWindow;
}

class MainWindow : public QMainWindow{
    Q_OBJECT

public:
    explicit MainWindow(QWidget *parent = 0);
    void StereoVisionProcessInit();
    void printHelp();
    void openStereoSource(int inputNum);
    void createTrackbars();
    QImage putImage(const Mat& mat);
    ~MainWindow();

private:
    Ui::MainWindow *ui;
    StereoProcessor *stereo;
    SetStereoParams *stereoParamsSetupWindow;

    QImage qimageL,qimageR;

    QTimer* tmrTimer;

public slots:
    void StereoVisionProcessAndUpdateGUI();

private slots:
    void on_btnPauseOrResume_clicked();
    void on_btnShowDisparityMap_clicked();
    void on_btnShowStereoParamSetup_clicked();
    void on_btnShow3DReconstruction_clicked();
    void on_btnShowInputImages_clicked();
    void on_btnShowTrackingObjectView_clicked();
    void on_btnShowDiffImage_clicked();
    void on_btnShowDiffImage_2_clicked();
};

#endif // MAINWINDOW_H
\end{lstlisting}

\textbf{reprojectImageTo3D.h}
\begin{lstlisting}
/*
 * reprojectImageTo3D.h
 *
 *  Created on: Jun 18, 2015
 *      Author: nicolasrosa
 */

#ifndef reproject_Image_To_3D_LIB_H_
#define reproject_Image_To_3D_LIB_H_

/* Libraries */
#include <opencv2/opencv.hpp>
#include <fstream>

using namespace cv;

/* Calibration */
#define RESOLUTION_640x480
//#define RESOLUTION_1280x720
#define CALIBRATION_ON

/* Functions Scope */
void on_trackbar(int,void*);

void imageProcessing1(Mat img, Mat imgMedian, Mat imgMedianBGR);
void imageProcessing2(Mat src, Mat imgE, Mat imgED,Mat cameraFeedL,bool isTrackingObjects);

void resizeFrames(Mat* frame1,Mat* frame2);
void changeResolution(VideoCapture* cap_l,VideoCapture* cap_r);
void contrast_and_brightness(Mat &left,Mat &right,float alpha,float beta);

/* Global Variables */
bool isVideoFile=false;
bool isImageFile=false;
bool needCalibration=false;
bool isStereoParamSetupTrackbarsCreated=false;
bool isTrackingObjects=true;

#endif /* reproject_Image_To_3D_LIB_H_ */
\end{lstlisting}

\textbf{setstereoparams.h}
\begin{lstlisting}
/*
 * setstereoparams.h
 *
 *  Created on: Oct 1, 2015
 *      Author: nicolasrosa
 */

#ifndef SETSTEREOPARAMS_H
#define SETSTEREOPARAMS_H

#include <QDialog>

class StereoProcessor; // forward-declaration

namespace Ui {
    class SetStereoParams;
}

class SetStereoParams : public QDialog{
    Q_OBJECT

public:
    explicit SetStereoParams(QWidget *parent = 0, StereoProcessor *stereo = 0);
    void loadStereoParamsUi(int preFilterSize,int preFilterCap,int SADWindowSize,int minDisparity,int numberOfDisparities,int textureThreshold,int uniquenessRatio, int speckleWindowSize, int speckleRange,int disp12MaxDiff);

    ~SetStereoParams();

    bool isAlreadyShowing;

signals:
    void valuesChanged(int preFilterSize, int preFilterCap, int sadWindowSize, int minDisparity, int numOfDisparities, int textureThreshold, int uniquenessRatio, int speckleWindowSize, int speckleWindowRange, int disp12MaxDiff);
private slots:
    /* Sliders */
    void on_preFilterSize_slider_valueChanged(int value);
    void on_preFilterCap_slider_valueChanged(int value);
    void on_SADWindowSize_slider_valueChanged(int value);
    void on_minDisparity_slider_valueChanged(int value);
    void on_numberOfDisparities_slider_valueChanged(int value);
    void on_textureThreshold_slider_valueChanged(int value);
    void on_uniquenessRatio_slider_valueChanged(int value);
    void on_speckleWindowSize_slider_valueChanged(int value);
    void on_speckleRange_slider_valueChanged(int value);
    void on_disp12MaxDiff_slider_valueChanged(int value);

    /* SpinBoxes */
    void on_preFilterSize_spinBox_valueChanged(int value);
    void on_preFilterCap_spinBox_valueChanged(int value);
    void on_SADWindowSize_spinBox_valueChanged(int value);
    void on_minDisparity_spinBox_valueChanged(int value);
    void on_numberOfDisparities_spinBox_valueChanged(int value);
    void on_textureThreshold_spinBox_valueChanged(int value);
    void on_uniquenessRatio_spinBox_valueChanged(int value);
    void on_speckleWindowSize_spinBox_valueChanged(int value);
    void on_speckleRange_spinBox_valueChanged(int value);
    void on_disp12MaxDiff_spinBox_valueChanged(int value);

    void on_buttonBox_accepted();
    void on_buttonBox_rejected();

private:
    Ui::SetStereoParams *ui;
    StereoProcessor *stereo;

    void updateValues();
};

#endif // SETSTEREOPARAMS_H
\end{lstlisting}

\textbf{StereoCalib.h}
\begin{lstlisting}
/*
 * StereoCalib.h
 *
 *  Created on: Dec 3, 2015
 *      Author: nicolasrosa
 */

#ifndef STEREOCALIB_H
#define STEREOCALIB_H

/* Libraries */
#include <string>
#include <opencv2/opencv.hpp>

using namespace cv;
using namespace std;

class StereoCalib{
public:
    StereoCalib(); //Constructor
    void readQMatrix();
    void calculateQMatrix();
    void createKMatrix();

    string intrinsicsFileName;
    string extrinsicsFileName;
    string QmatrixFileName;
    string StereoParamFileName;

    Point2d imageCenter;

    Mat K,Q;
    double focalLength;
    double baseline;
    bool is320x240;
    bool is640x480;
    bool is1280x720;

    Mat M1,D1,M2,D2;
    Mat R,T,R1,P1,R2,P2;
    Rect roi1, roi2;
    bool isKcreated;
};

#endif // STEREOCALIB_H
\end{lstlisting}

\textbf{StereoConfig.h}
\begin{lstlisting}
/*
 * StereoConfig.h
 *
 *  Created on: Dec 3, 2015
 *      Author: nicolasrosa
 */

#ifndef STEREOCONFIG_H
#define STEREOCONFIG_H

class StereoConfig{
public:
    StereoConfig(); //Constructor
    //StereoConfig getConfig();

    int preFilterSize;
    int preFilterCap;
    int SADWindowSize;
    int minDisparity;
    int numberOfDisparities;
    int textureThreshold;
    int uniquenessRatio;
    int speckleWindowSize;
    int speckleRange;
    int disp12MaxDiff;
};

#endif // STEREOCONFIG_H
\end{lstlisting}

\textbf{StereoCustom.h}
\begin{lstlisting}
/*
 * StereoCustom.h
 *
 *  Created on: Dec 3, 2015
 *      Author: nicolasrosa
 */

#ifndef STEREOCUSTOM_H
#define STEREOCUSTOM_H

/* Libraries */
#include <string>

using namespace std;

/* Global Variables */
const string trackbarWindowName = "Stereo Param Setup";
//bool isVideoFile=false,isImageFile=false,needCalibration=false,isStereoParamSetupTrackbarsCreated=false,isTrackingObjects=true;;

/* Threshold, Erosion, Dilation and Blur Constants */
#define THRESH_VALUE   100
#define EROSION_SIZE     5
#define DILATION_SIZE    5
#define BLUR_SIZE        3

/* Trackbars Variables
 * Initial min and max BM Parameters values.These will be changed using trackbars
 */
const int preFilterSize_MAX		 	= 100;
const int preFilterCap_MAX		 	= 100;
const int SADWindowSize_MAX		 	= 100;
const int minDisparity_MAX		 	= 100;
const int numberOfDisparities_MAX 	= 16;
const int textureThreshold_MAX		= 100;
const int uniquenessRatio_MAX		= 100;
const int speckleWindowSize_MAX	 	= 100;
const int speckleRange_MAX		 	= 100;
const int disp12MaxDiff_MAX		 	= 1;

#endif // STEREOCUSTOM_H
\end{lstlisting}

\textbf{StereoDiff.h}
\begin{lstlisting}
/*
 * StereoDiff.h
 *
 *  Created on: Dec 3, 2015
 *      Author: nicolasrosa
 */

#ifndef STEREODIFF_H
#define STEREODIFF_H

/* Libraries */
#include <opencv2/opencv.hpp>

using namespace cv;
using namespace std;

class StereoDiff{
public:
    StereoDiff(); //Constructor
    void createDiffImage(Mat,Mat);
    void createResAND(Mat,Mat);
    void convertToBGR();
    void addRedLines();

    bool StartDiff;
    Mat diffImage;

    Mat res_AND;
    Mat imageL;
    Mat res_AND_BGR;
    Mat res_AND_BGR_channels[3];

    double alpha;
    double beta;
    Mat res_ADD;
};

#endif // STEREODIFF_H
\end{lstlisting}

\textbf{StereoDisparityMap.h}
\begin{lstlisting}
/*
 * StereoDisparityMap.h
 *
 *  Created on: Dec 3, 2015
 *      Author: nicolasrosa
 */

#ifndef STEREODISPARITYMAP_H
#define STEREODISPARITYMAP_H

/* Libraries */
#include <opencv2/opencv.hpp>

using namespace cv;
using namespace std;

class StereoDisparityMap{
public:
    StereoDisparityMap(); //Constructor

    Mat disp_16S;
    Mat disp_8U;
    Mat disp_BGR;
};

#endif // STEREODISPARITYMAP_H
\end{lstlisting}

\textbf{StereoFlags.h}
\begin{lstlisting}
/*
 * StereoFlags.h
 *
 *  Created on: Dec 3, 2015
 *      Author: nicolasrosa
 */

#ifndef STEREOFLAGS_H
#define STEREOFLAGS_H

class StereoFlags{
public:
    StereoFlags(); //Constructor

    bool showInputImages;
    bool showXYZ;
    bool showStereoParam;
    bool showStereoParamValues;
    bool showFPS;
    bool showDisparityMap;
    bool show3Dreconstruction;
    bool showTrackingObjectView;
    bool showDiffImage;
    bool showWarningLines;
};

#endif // STEREOFLAGS_H
\end{lstlisting}

\textbf{StereoProcessor.h}
\begin{lstlisting}
/*
 * StereoProcessor.h
 *
 *  Created on: Oct 20, 2015
 *      Author: nicolasrosa
 */

#ifndef STEREOPROCESSOR_H
#define STEREOPROCESSOR_H

/* Libraries */
#include <opencv2/opencv.hpp>

/* Custom Libraries */
#include "StereoCalib.h"
#include "StereoCustom.h"
#include "StereoConfig.h"
#include "StereoDiff.h"
#include "StereoDisparityMap.h"
#include "StereoFlags.h"

#include "3DReconstruction.h"

using namespace cv;
using namespace std;

class StereoProcessor : public StereoConfig{
public:
    StereoProcessor(int inputNum); //Constructor
    int getInputNum();

    void readConfigFile();
    void readStereoConfigFile();

    void stereoInit();
    void stereoCalib();
    void setStereoParams();
    void setValues(int preFilterSize, int preFilterCap, int sadWindowSize, int minDisparity, int numOfDisparities, int textureThreshold, int uniquenessRatio, int speckleWindowSize, int speckleWindowRange, int disp12MaxDiff);

    void imageProcessing(Mat src, Mat imgE, Mat imgED,Mat trackingView,bool isTrackingObjects);

    void saveLastFrames();

    Mat imageL[2],imageR[2];
    Mat	imageL_grey[2],imageR_grey[2];
    VideoCapture capL,capR;

    Ptr<StereoBM> bm;
    StereoCalib calib;
    StereoConfig stereocfg;
    StereoDisparityMap disp;
    Reconstruction3D view3D;
    StereoDiff diff;
    StereoFlags flags;
    Size imageSize;
    int numRows;

    /* Results */
    Mat imgThreshold;
    Mat trackingView;

    bool showStereoParamsValues;

private:
    int inputNum;
};

#endif // STEREOPROCESSOR_H
\end{lstlisting}

\subsection{C�digos-Fonte}

\textbf{3DReconstruction.cpp}
\begin{lstlisting}
/*
 * 3DReconstruction.cpp
 *
 *  Created on: Oct 25, 2015
 *      Author: nicolasrosa
 */

#include "3DReconstruction.h"

/* Constructor */
Reconstruction3D::Reconstruction3D(){}

void Reconstruction3D::setViewPoint(double x,double y,double z){
    this->viewpoint.x = x;
    this->viewpoint.y = y;
    this->viewpoint.z = z;
}

void Reconstruction3D::setLookAtPoint(double x,double y,double z){
    this->lookatpoint.x = x;
    this->lookatpoint.y = y;
    this->lookatpoint.z = z;
}

void Reconstruction3D::PointCloudInit(double baseline,bool isSub){
    this->dist=Mat::zeros(5,1,CV_64F);
    this->Rotation=Mat::eye(3,3,CV_64F);
    this->t=Mat::zeros(3,1,CV_64F);

    this->isSub = isSub;
    this->step = baseline/10;
}

void Reconstruction3D::eular2rot(double yaw,double pitch, double roll,Mat& dest){
    double theta = yaw/180.0*CV_PI;
    double pusai = pitch/180.0*CV_PI;
    double phi = roll/180.0*CV_PI;

    double datax[3][3] = {{1.0,0.0,0.0},{0.0,cos(theta),-sin(theta)},{0.0,sin(theta),cos(theta)}};
    double datay[3][3] = {{cos(pusai),0.0,sin(pusai)},{0.0,1.0,0.0},{-sin(pusai),0.0,cos(pusai)}};
    double dataz[3][3] = {{cos(phi),-sin(phi),0.0},{sin(phi),cos(phi),0.0},{0.0,0.0,1.0}};
    Mat Rx(3,3,CV_64F,datax);
    Mat Ry(3,3,CV_64F,datay);
    Mat Rz(3,3,CV_64F,dataz);
    Mat rr=Rz*Rx*Ry;
    rr.copyTo(dest);
}

void Reconstruction3D::lookat(Point3d from, Point3d to, Mat& destR){
    double x=(to.x-from.x);
    double y=(to.y-from.y);
    double z=(to.z-from.z);

    double pitch =asin(x/sqrt(x*x+z*z))/CV_PI*180.0;
    double yaw =asin(-y/sqrt(y*y+z*z))/CV_PI*180.0;

    eular2rot(yaw, pitch, 0,destR);
}

void Reconstruction3D::projectImagefromXYZ(Mat &image, Mat &destimage, Mat &disp, Mat &destdisp, Mat &xyz, Mat &R, Mat &t, Mat &K, Mat &dist, bool isSub){
    Mat mask;
    if(mask.empty())mask=Mat::zeros(image.size(),CV_8U);
    if(disp.type()==CV_8U){
        projectImagefromXYZ_<unsigned char>(image,destimage, disp, destdisp, xyz, R, t, K, dist, mask,isSub);
    }
    else if(disp.type()==CV_16S){
        projectImagefromXYZ_<short>(image,destimage, disp, destdisp, xyz, R, t, K, dist, mask,isSub);
    }
    else if(disp.type()==CV_16U){
        projectImagefromXYZ_<unsigned short>(image,destimage, disp, destdisp, xyz, R, t, K, dist, mask,isSub);
    }
    else if(disp.type()==CV_32F){
        projectImagefromXYZ_<float>(image,destimage, disp, destdisp, xyz, R, t, K, dist, mask,isSub);
    }
    else if(disp.type()==CV_64F){
        projectImagefromXYZ_<double>(image,destimage, disp, destdisp, xyz, R, t, K, dist, mask,isSub);
    }
}

template <class T>
static void fillOcclusionInv_(Mat& src, T invalidvalue){
    int bb=1;
    const int MAX_LENGTH=src.cols*0.8;
    //#pragma omp parallel for
    for(int j=bb;j<src.rows-bb;j++){
        T* s = src.ptr<T>(j);
        //const T st = s[0];
        //const T ed = s[src.cols-1];
        s[0]=0;
        s[src.cols-1]=0;
        for(int i=0;i<src.cols;i++){
            if(s[i]==invalidvalue){
                int t=i;
                do{
                    t++;
                    if(t>src.cols-1)break;
                }while(s[t]==invalidvalue);

                const T dd = max(s[i-1],s[t]);
                if(t-i>MAX_LENGTH){
                    for(int n=0;n<src.cols;n++){
                        s[n]=invalidvalue;
                    }
                }
                else{
                    for(;i<t;i++){
                        s[i]=dd;
                    }
                }
            }
        }
    }
}

template <class T>
static void projectImagefromXYZ_(Mat& image, Mat& destimage, Mat& disp, Mat& destdisp, Mat& xyz, Mat& R, Mat& t, Mat& K, Mat& dist, Mat& mask, bool isSub){
    if(destimage.empty())destimage=Mat::zeros(Size(image.size()),image.type());
    if(destdisp.empty())destdisp=Mat::zeros(Size(image.size()),disp.type());

    vector<Point2f> pt;
    if(dist.empty()) dist = Mat::zeros(Size(5,1),CV_32F);
    cv::projectPoints(xyz,R,t,K,dist,pt);
    destimage.setTo(0);
    destdisp.setTo(0);

    //#pragma omp parallel for
    for(int j=1;j<image.rows-1;j++){
        int count=j*image.cols;
        uchar* img=image.ptr<uchar>(j);
        uchar* m=mask.ptr<uchar>(j);
        for(int i=0;i<image.cols;i++,count++){
            int x=(int)(pt[count].x+0.5);
            int y=(int)(pt[count].y+0.5);
            if(m[i]==255)continue;
            if(pt[count].x>=1 && pt[count].x<image.cols-1 && pt[count].y>=1 && pt[count].y<image.rows-1){
                short v=destdisp.at<T>(y,x);
                if(v<disp.at<T>(j,i)){
                    destimage.at<uchar>(y,3*x+0)=img[3*i+0];
                    destimage.at<uchar>(y,3*x+1)=img[3*i+1];
                    destimage.at<uchar>(y,3*x+2)=img[3*i+2];
                    destdisp.at<T>(y,x)=disp.at<T>(j,i);

                    if(isSub){
                        if((int)pt[count+image.cols].y-y>1 && (int)pt[count+1].x-x>1){
                            destimage.at<uchar>(y,3*x+3)=img[3*i+0];
                            destimage.at<uchar>(y,3*x+4)=img[3*i+1];
                            destimage.at<uchar>(y,3*x+5)=img[3*i+2];

                            destimage.at<uchar>(y+1,3*x+0)=img[3*i+0];
                            destimage.at<uchar>(y+1,3*x+1)=img[3*i+1];
                            destimage.at<uchar>(y+1,3*x+2)=img[3*i+2];

                            destimage.at<uchar>(y+1,3*x+3)=img[3*i+0];
                            destimage.at<uchar>(y+1,3*x+4)=img[3*i+1];
                            destimage.at<uchar>(y+1,3*x+5)=img[3*i+2];

                            destdisp.at<T>(y,x+1)=disp.at<T>(j,i);
                            destdisp.at<T>(y+1,x)=disp.at<T>(j,i);
                            destdisp.at<T>(y+1,x+1)=disp.at<T>(j,i);
                        }
                        else if((int)pt[count-image.cols].y-y<-1 && (int)pt[count-1].x-x<-1){
                            destimage.at<uchar>(y,3*x-3)=img[3*i+0];
                            destimage.at<uchar>(y,3*x-2)=img[3*i+1];
                            destimage.at<uchar>(y,3*x-1)=img[3*i+2];

                            destimage.at<uchar>(y-1,3*x+0)=img[3*i+0];
                            destimage.at<uchar>(y-1,3*x+1)=img[3*i+1];
                            destimage.at<uchar>(y-1,3*x+2)=img[3*i+2];

                            destimage.at<uchar>(y-1,3*x-3)=img[3*i+0];
                            destimage.at<uchar>(y-1,3*x-2)=img[3*i+1];
                            destimage.at<uchar>(y-1,3*x-1)=img[3*i+2];

                            destdisp.at<T>(y,x-1)=disp.at<T>(j,i);
                            destdisp.at<T>(y-1,x)=disp.at<T>(j,i);
                            destdisp.at<T>(y-1,x-1)=disp.at<T>(j,i);
                        }
                        else if((int)pt[count+1].x-x>1){
                            destimage.at<uchar>(y,3*x+3)=img[3*i+0];
                            destimage.at<uchar>(y,3*x+4)=img[3*i+1];
                            destimage.at<uchar>(y,3*x+5)=img[3*i+2];

                            destdisp.at<T>(y,x+1)=disp.at<T>(j,i);
                        }
                        else if((int)pt[count-1].x-x<-1){
                            destimage.at<uchar>(y,3*x-3)=img[3*i+0];
                            destimage.at<uchar>(y,3*x-2)=img[3*i+1];
                            destimage.at<uchar>(y,3*x-1)=img[3*i+2];

                            destdisp.at<T>(y,x-1)=disp.at<T>(j,i);
                        }
                        else if((int)pt[count+image.cols].y-y>1){
                            destimage.at<uchar>(y+1,3*x+0)=img[3*i+0];
                            destimage.at<uchar>(y+1,3*x+1)=img[3*i+1];
                            destimage.at<uchar>(y+1,3*x+2)=img[3*i+2];

                            destdisp.at<T>(y+1,x)=disp.at<T>(j,i);
                        }
                        else if((int)pt[count-image.cols].y-y<-1){
                            destimage.at<uchar>(y-1,3*x+0)=img[3*i+0];
                            destimage.at<uchar>(y-1,3*x+1)=img[3*i+1];
                            destimage.at<uchar>(y-1,3*x+2)=img[3*i+2];

                            destdisp.at<T>(y-1,x)=disp.at<T>(j,i);
                        }
                    }
                }
            }
        }
    }

    if(isSub)
    {
        Mat image2;
        Mat disp2;
        destimage.copyTo(image2);
        destdisp.copyTo(disp2);
        const int BS=1;
        //#pragma omp parallel for
        for(int j=BS;j<image.rows-BS;j++){
            uchar* img=destimage.ptr<uchar>(j);
            T* m = disp2.ptr<T>(j);
            T* dp = destdisp.ptr<T>(j);
            for(int i=BS;i<image.cols-BS;i++){
                if(m[i]==0){
                    int count=0;
                    int d=0;
                    int r=0;
                    int g=0;
                    int b=0;
                    for(int l=-BS;l<=BS;l++){
                        T* dp2 = disp2.ptr<T>(j+l);
                        uchar* imageR = image2.ptr<uchar>(j+l);
                        for(int k=-BS;k<=BS;k++){
                            if(dp2[i+k]!=0){
                                count++;
                                d+=dp2[i+k];
                                r+=imageR[3*(i+k)+0];
                                g+=imageR[3*(i+k)+1];
                                b+=imageR[3*(i+k)+2];
                            }
                        }
                    }
                    if(count!=0){
                        double div = 1.0/count;
                        dp[i]=d*div;
                        img[3*i+0]=r*div;
                        img[3*i+1]=g*div;
                        img[3*i+2]=b*div;
                    }
                }
            }
        }
    }
}

void fillOcclusion(Mat& src, int invalidvalue, bool isInv){
    if(isInv){
        if(src.type()==CV_8U){
            fillOcclusionInv_<uchar>(src, (uchar)invalidvalue);
        }
        else if(src.type()==CV_16S){
            fillOcclusionInv_<short>(src, (short)invalidvalue);
        }
        else if(src.type()==CV_16U){
            fillOcclusionInv_<unsigned short>(src, (unsigned short)invalidvalue);
        }
        else if(src.type()==CV_32F){
            fillOcclusionInv_<float>(src, (float)invalidvalue);
        }
    }
    else{
        if(src.type()==CV_8U){
            fillOcclusion_<uchar>(src, (uchar)invalidvalue);
        }
        else if(src.type()==CV_16S){
            fillOcclusion_<short>(src, (short)invalidvalue);
        }
        else if(src.type()==CV_16U){
            fillOcclusion_<unsigned short>(src, (unsigned short)invalidvalue);
        }
        else if(src.type()==CV_32F){
            fillOcclusion_<float>(src, (float)invalidvalue);
        }
    }
}

template <class T>
static void fillOcclusion_(Mat& src, T invalidvalue){
    int bb=1;
    const int MAX_LENGTH=src.cols*0.5;
    //#pragma omp parallel for
    for(int j=bb;j<src.rows-bb;j++){
        T* s = src.ptr<T>(j);
        //const T st = s[0];
        //const T ed = s[src.cols-1];
        s[0]=255;
        s[src.cols-1]=255;
        for(int i=0;i<src.cols;i++){
            if(s[i]<=invalidvalue){
                int t=i;
                do{
                    t++;
                    if(t>src.cols-1)break;
                }while(s[t]<=invalidvalue);

                const T dd = min(s[i-1],s[t]);
                if(t-i>MAX_LENGTH){
                    for(int n=0;n<src.cols;n++){
                        s[n]=invalidvalue;
                    }
                }
                else{
                    for(;i<t;i++){
                        s[i]=dd;
                    }
                }
            }
        }
    }
}
\end{lstlisting}

\textbf{main.cpp}
\begin{lstlisting}
#include "mainwindow.h"

#include <QApplication>

//#include <QHBoxLayout>
//#include <QSlider>
//#include <QSpinBox>

int main(int argc, char *argv[]){
    QApplication app(argc, argv);
    MainWindow mainwindow;

    mainwindow.show();

//    QWidget *window = new QWidget;
//    window->setWindowTitle("Enter Your Age");

//    QSpinBox *spinBox = new QSpinBox;
//    QSlider *slider = new QSlider(Qt::Horizontal);
//    spinBox->setRange(0, 130);
//    slider->setRange(0, 130);

//    QObject::connect(spinBox, SIGNAL(valueChanged(int)),slider, SLOT(setValue(int)));
//    QObject::connect(slider, SIGNAL(valueChanged(int)),spinBox, SLOT(setValue(int)));
//    spinBox->setValue(35);

//    QHBoxLayout *layout = new QHBoxLayout;

//    layout->addWidget(spinBox);
//    layout->addWidget(slider);
//    window->setLayout(layout);

//    window->show();

    return app.exec();
}
\end{lstlisting}

\textbf{mainwindow.cpp}
\begin{lstlisting}
/* Project: reprojectImageTo3D - BlockMatching Algorithm
 * mainwindow.cpp
 *
 *  Created on: June, 2015
 *      Author: nicolasrosa
 *
 * // Credits: http://opencv.jp/opencv2-x-samples/point-cloud-rendering
 * // Credits: Kyle Hounslow - https://www.youtube.com/watch?v=bSeFrPrqZ2A
 */

/* Libraries */
#include <QtCore>
#include <opencv2/imgproc/imgproc.hpp>

/* Custom Libraries */
#include "reprojectImageTo3D.h"
#include "mainwindow.h"
#include "ui_mainwindow.h"

using namespace cv;
using namespace std;

void writeMatToFile(cv::Mat& m, const char* filename);

//MainWindow::MainWindow(QWidget *parent):QMainWindow(parent),ui(new Ui::MainWindow),stereo(new StereoProcessor(6)){
MainWindow::MainWindow(QWidget *parent):QMainWindow(parent),ui(new Ui::MainWindow){
    ui->setupUi(this);

    this->stereo = new StereoProcessor(1);
    StereoVisionProcessInit();

    tmrTimer = new QTimer(this);
    connect(tmrTimer,SIGNAL(timeout()),this,SLOT(StereoVisionProcessAndUpdateGUI()));
    tmrTimer->start(20);
}

MainWindow::~MainWindow(){
    delete ui;
}

void MainWindow::on_btnPauseOrResume_clicked(){
    if(tmrTimer->isActive() == true){
        tmrTimer->stop();
        cout << "Paused!" << endl;
        ui->btnPauseOrResume->setText("Resume");
    }else{
        tmrTimer->start(20);
        cout << "Resumed!" << endl;
        ui->btnPauseOrResume->setText("Pause");
    }
}

void MainWindow::on_btnShowStereoParamSetup_clicked(){
    stereoParamsSetupWindow = new SetStereoParams(this, stereo);

    cout << "[Stereo Param Setup] Stereo Parameters Configuration Loaded!" << endl;
    this->stereoParamsSetupWindow->loadStereoParamsUi(stereo->stereocfg.preFilterSize,
                                                      stereo->stereocfg.preFilterCap,
                                                      stereo->stereocfg.SADWindowSize,
                                                      stereo->stereocfg.minDisparity,
                                                      stereo->stereocfg.numberOfDisparities,
                                                      stereo->stereocfg.textureThreshold,
                                                      stereo->stereocfg.uniquenessRatio,
                                                      stereo->stereocfg.speckleWindowSize,
                                                      stereo->stereocfg.speckleRange,
                                                      stereo->stereocfg.disp12MaxDiff);
    stereoParamsSetupWindow->show();
}

void MainWindow::StereoVisionProcessInit(){
    cerr << "Arrumar a Matrix K, os valores das �ltimas colunas est�o errados." << endl;
    cerr << "Arrumar a fun��o StereoProcessor::calculateQMatrix()." << endl;
    cerr << "Arrumar o Constructor da classe StereoDisparityMap para Aloca��o de Mem�ria das vari�veis: disp_16S,disp_8U,disp_BGR" << endl;
    cerr << "Arrumar o tipo de execu��o da Stereo Param Setup, fazer com que a execu��o da main n�o pause." << endl;
    cerr << "Arrumar a funcionalidade do Bot�o Pause/Resume, n�o est� funcionando." << endl;

    printHelp();

    //(1) Open Image Source
    openStereoSource(stereo->getInputNum());
    stereo->readConfigFile();
    stereo->readStereoConfigFile();

    //(2) Camera Setting

    // Checking Resolution
    stereo->calib.is320x240  = false;
    stereo->calib.is640x480  = true;
    stereo->calib.is1280x720 = false;

    if(isVideoFile){
        stereo->imageSize.width = stereo->capL.get(CV_CAP_PROP_FRAME_WIDTH);
        stereo->imageSize.height = stereo->capL.get(CV_CAP_PROP_FRAME_HEIGHT);
    }else{
        stereo->imageSize.width = stereo->imageL[0].cols;
        stereo->imageSize.height = stereo->imageL[0].rows;
    }

    if(stereo->imageSize.width==0 && stereo->imageSize.height==0){
        cerr << "Number of Cols and Number of Rows equal to ZERO!" << endl;
    }else{
        cout << "Input Resolution(Width,Height): (" << stereo->imageSize.width << "," << stereo->imageSize.height << ")" << endl << endl;
    }

    //(3) Stereo Initialization
    stereo->bm = StereoBM::create(16,9);
    stereo->stereoInit();

    //(4) Stereo Calibration
    if(needCalibration){
        cout << "Calibration: ON" << endl;
        stereo->stereoCalib();

        // Compute the Q Matrix
        stereo->calib.readQMatrix(); //true=640x480 false=others

        //Point2d imageCenter = Point2d((imageL[0].cols-1.0)/2.0,(imageL[0].rows-1.0)/2.0);
        //calculateQMatrix();

        // Compute the K Matrix
        ////        // Checking Intrinsic Matrix
        ////        if(stereo->calib.isKcreated){
        ////           cout << "The Intrinsic Matrix is already Created." << endl << endl;
        ////        }else{
        //            //createKMatrix();
        // //       }
        stereo->calib.createKMatrix();

    }else{
        cout << "Calibration: OFF" << endl << endl;
        cerr << "Warning: Couldn't generate 3D Reconstruction. Please, check Q,K Matrix." << endl;

        //stereo->readQMatrix(); //true=640x480 false=others
        //stereo->createKMatrix();
    }

    //Setting StereoBM Parameters
    stereo->setStereoParams();

    //(5) Point Cloud Initialization
    stereo->view3D.PointCloudInit(stereo->calib.baseline/10,true);

    stereo->view3D.setViewPoint(20.0,20.0,-stereo->calib.baseline*10);
    stereo->view3D.setLookAtPoint(22.0,16.0,stereo->calib.baseline*10.0);

}

void MainWindow::StereoVisionProcessAndUpdateGUI(){
    //Local Variables
    char key=0;

    //Timing
    int frameCounter=0;
    int fps,lastTime = clock();

    //(6) Rendering Loop
    while(key!='q'){
        if(isVideoFile){
            stereo->capL >> stereo->imageL[0];
            stereo->capR >> stereo->imageR[0];

            resizeFrames(&stereo->imageL[0],&stereo->imageR[0]);

            if(needCalibration){
                stereo->imageSize = stereo->imageL[0].size();
                stereoRectify(stereo->calib.M1,stereo->calib.D1,stereo->calib.M2,stereo->calib.D2,stereo->imageSize,stereo->calib.R,stereo->calib.T,stereo->calib.R1,stereo->calib.R2,stereo->calib.P1,stereo->calib.P2,stereo->calib.Q,CALIB_ZERO_DISPARITY,-1,stereo->imageSize,&stereo->calib.roi1,&stereo->calib.roi2);
                Mat rmap[2][2];

                initUndistortRectifyMap(stereo->calib.M1, stereo->calib.D1, stereo->calib.R1, stereo->calib.P1, stereo->imageSize, CV_16SC2, rmap[0][0], rmap[0][1]);
                initUndistortRectifyMap(stereo->calib.M2, stereo->calib.D2, stereo->calib.R2, stereo->calib.P2, stereo->imageSize, CV_16SC2, rmap[1][0], rmap[1][1]);
                Mat imageLr, imageRr;
                remap(stereo->imageL[0], imageLr, rmap[0][0], rmap[0][1], INTER_LINEAR);
                remap(stereo->imageR[0], imageRr, rmap[1][0], rmap[1][1], INTER_LINEAR);

                stereo->imageL[0] = imageLr;
                stereo->imageR[0] = imageRr;
            }
        }

        //Setting StereoBM Parameters
        //stereo->setStereoParams();

        // Convert BGR to Grey Scale
        cvtColor(stereo->imageL[0],stereo->imageL_grey[0],CV_BGR2GRAY);
        cvtColor(stereo->imageR[0],stereo->imageR_grey[0],CV_BGR2GRAY);

        stereo->bm->compute(stereo->imageL_grey[0],stereo->imageR_grey[0],stereo->disp.disp_16S);
        //fillOcclusion(disp,16,false);

        normalize(stereo->disp.disp_16S, stereo->disp.disp_8U, 0, 255, CV_MINMAX, CV_8U);
        applyColorMap(stereo->disp.disp_8U,stereo->disp.disp_BGR, COLORMAP_JET);

        /* Image Processing */
        Mat disp_8Ueroded;Mat disp_8U_eroded_dilated;

        stereo->imageProcessing(stereo->disp.disp_8U,disp_8Ueroded,disp_8U_eroded_dilated,stereo->imageL[0],true);

        //(7) Projecting 3D point cloud to image
        if(stereo->flags.show3Dreconstruction){
            cv::reprojectImageTo3D(stereo->disp.disp_16S,stereo->view3D.depth,stereo->calib.Q);
            Mat xyz = stereo->view3D.depth.reshape(3,stereo->view3D.depth.size().area());

            stereo->view3D.lookat(stereo->view3D.viewpoint, stereo->view3D.lookatpoint , stereo->view3D.Rotation);
            stereo->view3D.t.at<double>(0,0)=stereo->view3D.viewpoint.x;
            stereo->view3D.t.at<double>(1,0)=stereo->view3D.viewpoint.y;
            stereo->view3D.t.at<double>(2,0)=stereo->view3D.viewpoint.z;

            if(stereo->flags.showXYZ){
                //cout<< stereo->view3D.t <<endl;
                cout << "x: " << stereo->view3D.t.at<double>(0,0) << endl;
                cout << "y: " << stereo->view3D.t.at<double>(1,0) << endl;
                cout << "z: " << stereo->view3D.t.at<double>(2,0) << endl;
            }

            stereo->view3D.t=stereo->view3D.Rotation*stereo->view3D.t;

            //projectImagefromXYZ(imageL[0],disp3Dviewer,disp,disp3D,xyz,Rotation,t,K,dist,isSub);
            stereo->view3D.projectImagefromXYZ(stereo->disp.disp_BGR,stereo->view3D.disp3D_BGR,stereo->disp.disp_16S,stereo->view3D.disp3D,xyz,stereo->view3D.Rotation,stereo->view3D.t,stereo->calib.K,stereo->view3D.dist,stereo->view3D.isSub);

            // GUI Output
            stereo->view3D.disp3D.convertTo(stereo->view3D.disp3D_8U,CV_8U,0.5);
            //imshow("3D Depth",disp3D);
            //imshow("3D Viewer",disp3Dviewer);
            //imshow("3D Depth RGB",disp3DBGR);

            QImage qimageL = putImage(stereo->view3D.disp3D_8U);
            QImage qimageR = putImage(stereo->view3D.disp3D_BGR);

            ui->lblOriginalLeft->setPixmap(QPixmap::fromImage(qimageL));
            ui->lblOriginalRight->setPixmap(QPixmap::fromImage(qimageR));
        }

        //(8)Movement Difference between Frames
        //if(stereo->diff.StartDiff){
        if(stereo->flags.showDiffImage || stereo->flags.showWarningLines){
            if(stereo->diff.StartDiff){
                stereo->diff.createDiffImage(stereo->imageL_grey[0],stereo->imageL_grey[1]);

                if(stereo->diff.diffImage.data){
                    stereo->diff.createResAND(stereo->diff.diffImage,stereo->imgThreshold);
                    stereo->diff.convertToBGR();
                    stereo->imageL[0].copyTo(stereo->diff.imageL);
                    stereo->diff.addRedLines();

                    //imshow("imgThreshold",stereo->imgThreshold);
                    //imshow("DiffImage",stereo->diff.diffImage);
                    //imshow("Bitwise_AND",stereo->diff.res_AND);
                }

                stereo->saveLastFrames();
            }else{
                stereo->saveLastFrames();
                stereo->diff.StartDiff=1;
            }
        }

        //(9)OpenCV and GUI Output
        if(stereo->flags.showInputImages){
            QImage qimageL = putImage(stereo->imageL[0]);
            QImage qimageR = putImage(stereo->imageR[0]);

            ui->lblOriginalLeft->setPixmap(QPixmap::fromImage(qimageL));
            ui->lblOriginalRight->setPixmap(QPixmap::fromImage(qimageR));
        }

        if(stereo->flags.showDisparityMap){
            QImage qimageL = putImage(stereo->disp.disp_8U);
            QImage qimageR = putImage(stereo->disp.disp_BGR);

            ui->lblOriginalLeft->setPixmap(QPixmap::fromImage(qimageL));
            ui->lblOriginalRight->setPixmap(QPixmap::fromImage(qimageR));
        }

        if(stereo->flags.showTrackingObjectView){
            QImage qimageL = putImage(stereo->trackingView);
            QImage qimageR = putImage(stereo->imgThreshold);

            ui->lblOriginalLeft->setPixmap(QPixmap::fromImage(qimageL));
            ui->lblOriginalRight->setPixmap(QPixmap::fromImage(qimageR));
        }

        if(stereo->flags.showDiffImage && stereo->diff.StartDiff){
            this->qimageL = putImage(stereo->diff.diffImage);
            this->qimageR = putImage(stereo->diff.res_AND);

            ui->lblOriginalLeft->setPixmap(QPixmap::fromImage(qimageL));
            ui->lblOriginalRight->setPixmap(QPixmap::fromImage(qimageR));
        }

        if(stereo->flags.showWarningLines && stereo->diff.StartDiff){
            this->qimageL = putImage(stereo->diff.res_ADD);
            this->qimageR = putImage(stereo->diff.res_AND_BGR);

            ui->lblOriginalLeft->setPixmap(QPixmap::fromImage(qimageL));
            ui->lblOriginalRight->setPixmap(QPixmap::fromImage(qimageR));
        }

        //       // if(showStereoParam && !isStereoParamSetupTrackbarsCreated){
        //       if(showStereoParam && !isStereoParamSetupTrackbarsCreated){
        //           isStereoParamSetupTrackbarsCreated=true;
        //           createTrackbars();
        //            cout << "oi" << endl;
        //        }else{
        //            destroyWindow(trackbarWindowName);
        //            isStereoParamSetupTrackbarsCreated=false;
        //        }

        //(10)Shortcuts
        key = waitKey(1);
        if(key=='`')
            printHelp();
        //        if(key=='1')
        //            showInputImages = !showInputImages;
        //        if(key=='2')
        //            showDisparityMap = !showDisparityMap;
        //        if(key=='3')
        //            show3Dreconstruction = !show3Dreconstruction;
        if(key=='4')
            stereo->flags.showXYZ = !stereo->flags.showXYZ;
        if(key=='5')
            stereo->flags.showFPS = !stereo->flags.showFPS;
        if(key=='6')
            stereo->flags.showStereoParamValues = !stereo->flags.showStereoParamValues;
        if(key=='7')
            stereo->flags.showDiffImage = !stereo->flags.showDiffImage;

        if(key=='f')
            stereo->view3D.isSub=stereo->view3D.isSub?false:true;
        if(key=='h')
            stereo->view3D.viewpoint.x+=stereo->view3D.step;
        if(key=='g')
            stereo->view3D.viewpoint.x-=stereo->view3D.step;
        if(key=='l')
            stereo->view3D.viewpoint.y+=stereo->view3D.step;
        if(key=='k')
            stereo->view3D.viewpoint.y-=stereo->view3D.step;
        if(key=='n')
            stereo->view3D.viewpoint.z+=stereo->view3D.step;
        if(key=='m')
            stereo->view3D.viewpoint.z-=stereo->view3D.step;

        if(key=='q')
            break;

        //(11)Video Loop - If the last frame is reached, reset the capture and the frameCounter
        frameCounter += 1;

        if(frameCounter == stereo->capR.get(CV_CAP_PROP_FRAME_COUNT)){
            frameCounter = 0;
            stereo->capL.set(CV_CAP_PROP_POS_FRAMES,0);
            stereo->capR.set(CV_CAP_PROP_POS_FRAMES,0);
        }

        if(1){
        //if(stereo->flags.showFPS){
            //cout << "Frames: " << frameCounter << "/" << capR.get(CV_CAP_PROP_FRAME_COUNT) << endl;
            //cout << "Current time(s): " << current_time << endl;
            //cout << "FPS: " << (frameCounter/current_time) << endl;
            fps = (int) (1000/((clock()/1000) - lastTime)); // time stuff
            lastTime = clock()/1000;
            //cout << clock() << endl;
            cout << "FPS: " << fps << endl;
        }
    }
    cout << "END" << endl;

    //return 0;
}

void MainWindow::printHelp(){
    //Console Output
    cout << "-----------------Help Menu-----------------\n"
         << "Run command line: ./reprojectImageTo3D\n"
         << "Keys:\n"
         << "'`' -\tShow Help\n"
         << "'1' -\tShow L/R Windows\t\t'4' -\tShow XYZ\n"
         << "'2' -\tShow Disparity Map\t\t'5' -\tShow FPS\n"
         << "'3' -\tShow 3D Reconstruction\t'6' -\tShow Stereo Parameters\n"
         << "\n3D Viewer Navigation:\n"
         << "x-axis:\t'g'/'h' -> +x,-x\n"
         << "y-axis:\t'l'/'k' -> +y,-y\n"
         << "z-axis:\t'n'/'m' -> +z,-z\n"
         << "-------------------------------------------\n"
         << "\n\n";

    //GUI
    ui->txtOutputBox->appendPlainText
            (QString("-----------------Help Menu-----------------\n")+
             QString("Run command line: ./reprojectImageTo3D\n")+
             QString("Keys:\n")+
             QString("'`' -\tShow Help\n")+
             QString("'1' -\tShow L/R Windows\t\t'4' -\tShow XYZ\n")+
             QString("'2' -\tShow Disparity Map\t\t'5' -\tShow FPS\n")+
             QString("'3' -\tShow 3D Reconstruction\t'6' -\tShow Stereo Parameters\n")+
             QString("\n3D Viewer Navigation:\n")+
             QString("x-axis:\t'g'/'h' -> +x,-x\n")+
             QString("y-axis:\t'l'/'k' -> +y,-y\n")+
             QString("z-axis:\t'n'/'m' -> +z,-z\n")+
             QString("-------------------------------------------\n")+
             QString("\n\n"));
}

void MainWindow::openStereoSource(int inputNum){
    string imageL_filename;
    string imageR_filename;

    // Create an object that decodes the input Video stream.
    cout << "Enter Video Number(1,2,3,4,5,6,7,8,9): " << endl;
    ui->txtOutputBox->appendPlainText(QString("Enter Video Number(1,2,3,4,5,6,7,8,9): "));
    //	scanf("%d",&inputNum);
    cout << "Input File: " << inputNum << endl;
    ui->txtOutputBox->appendPlainText(QString("Input File: ")+QString::number(inputNum));
    switch(inputNum){
    case 1:
        imageL_filename = "../../workspace/data/video10_l.avi";
        imageR_filename = "../../workspace/data/video10_r.avi";
        needCalibration=true;
        //ui->txtOutputBox->appendPlainText(QString("video2_denoised_long.avi"));
        break;
    case 2:
        imageL_filename = "../../workspace/data/video12_l.avi";
        imageR_filename = "../../workspace/data/video12_r.avi";
        needCalibration=true;
        //ui->txtOutputBox->appendPlainText(QString( "video0.avi"));
        break;
    case 3:
        imageL_filename = "../data/left/video1.avi";
        imageR_filename = "../data/right/video1.avi";
        needCalibration=true;
        //ui->txtOutputBox->appendPlainText(QString( "video1.avi"));
        break;
    case 4:
        imageL_filename = "../data/left/video2_noised.avi";
        imageR_filename = "../data/right/video2_noised.avi";
        needCalibration=true;
        //ui->txtOutputBox->appendPlainText(QString( "video2_noised.avi"));
        break;
    case 5:
        imageL_filename = "../data/left/20004.avi";
        imageR_filename = "../data/right/30004.avi";
        needCalibration=false;
        break;
    case 6:
        imageL_filename = "../../workspace/data/left/video15.avi";
        imageR_filename = "../../workspace/data/right/video15.avi";
        needCalibration=true;
        break;
    case 7:
        imageL_filename = "../../workspace/data/left/left1.png";
        imageR_filename = "../../workspace/data/right/right1.png";
        needCalibration=false;
        break;
    case 8:
        imageL_filename = "../data/left/left2.png";
        imageR_filename = "../data/right/right2.png";
        needCalibration=false;
        break;
    case 9:
        imageL_filename = "../data/left/left3.png";
        imageR_filename = "../data/right/right3.png";
        needCalibration=false;
        break;
    }

    if(imageL_filename.substr(imageL_filename.find_last_of(".") + 1) == "avi"){
        cout << "It's a Video file" << endl;
        ui->txtOutputBox->appendPlainText(QString("It's a Video file"));
        isVideoFile=true;

        stereo->capL.open(imageL_filename);
        stereo->capR.open(imageR_filename);

        if(!stereo->capL.isOpened() || !stereo->capR.isOpened()){		// Check if we succeeded
            cerr <<  "Could not open or find the input videos!" << endl ;
            ui->txtOutputBox->appendPlainText(QString( "Could not open or find the input videos!"));
            //return -1;
        }

        cout << "Input 1 Resolution: " << stereo->capR.get(CV_CAP_PROP_FRAME_WIDTH) << "x" << stereo->capR.get(CV_CAP_PROP_FRAME_HEIGHT) << endl;
        cout << "Input 2 Resolution: " << stereo->capL.get(CV_CAP_PROP_FRAME_WIDTH) << "x" << stereo->capL.get(CV_CAP_PROP_FRAME_HEIGHT) << endl << endl;
        ui->txtOutputBox->appendPlainText(QString("Input 1 Resolution: ") + QString::number(stereo->capL.get(CV_CAP_PROP_FRAME_WIDTH)) + QString("x") + QString::number(stereo->capL.get(CV_CAP_PROP_FRAME_HEIGHT)));
        ui->txtOutputBox->appendPlainText(QString("Input 2 Resolution: ") + QString::number(stereo->capR.get(CV_CAP_PROP_FRAME_WIDTH)) + QString("x") + QString::number(stereo->capR.get(CV_CAP_PROP_FRAME_HEIGHT)));
    }else{
        cout << "It is not a Video file" << endl;
        ui->txtOutputBox->appendPlainText(QString( "It is not a Video file"));
        if(imageL_filename.substr(imageL_filename.find_last_of(".") + 1) == "jpg" || imageL_filename.substr(imageL_filename.find_last_of(".") + 1) == "png"){
            cout << "It's a Image file" << endl;
            ui->txtOutputBox->appendPlainText(QString( "It's a Image file"));
            isImageFile=true;

            stereo->imageL[0] = imread(imageL_filename, CV_LOAD_IMAGE_COLOR);	// Read the file
            stereo->imageR[0] = imread(imageR_filename, CV_LOAD_IMAGE_COLOR);	// Read the file

            if(!stereo->imageL[0].data || !stereo->imageR[0].data){                     // Check for invalid input
                ui->txtOutputBox->appendPlainText(QString("Could not open or find the input images!"));
                return;
            }
        }else{
            cout << "It is not a Image file" << endl;
            ui->txtOutputBox->appendPlainText(QString( "It is not a Image file"));
        }
    }
}

void MainWindow::on_btnShowInputImages_clicked(){
    this->stereo->flags.showInputImages = true;
    this->stereo->flags.showDisparityMap = false;
    this->stereo->flags.show3Dreconstruction = false;
    this->stereo->flags.showTrackingObjectView = false;
    this->stereo->flags.showDiffImage = false;
    this->stereo->flags.showWarningLines = false;
}

void MainWindow::on_btnShowDisparityMap_clicked(){
    this->stereo->flags.showInputImages = false;
    this->stereo->flags.showDisparityMap = true;
    this->stereo->flags.show3Dreconstruction = false;
    this->stereo->flags.showTrackingObjectView = false;
    this->stereo->flags.showDiffImage = false;
    this->stereo->flags.showWarningLines = false;
}

void MainWindow::on_btnShow3DReconstruction_clicked(){
    this->stereo->flags.showInputImages = false;
    this->stereo->flags.showDisparityMap = false;
    this->stereo->flags.show3Dreconstruction = true;
    this->stereo->flags.showTrackingObjectView = false;
    this->stereo->flags.showDiffImage = false;
    this->stereo->flags.showWarningLines = false;
}

void MainWindow::on_btnShowTrackingObjectView_clicked(){
    this->stereo->flags.showInputImages = false;
    this->stereo->flags.showDisparityMap = false;
    this->stereo->flags.show3Dreconstruction = false;
    this->stereo->flags.showTrackingObjectView = true;
    this->stereo->flags.showDiffImage = false;
    this->stereo->flags.showWarningLines = false;
}

void MainWindow::on_btnShowDiffImage_clicked(){
    this->stereo->flags.showInputImages = false;
    this->stereo->flags.showDisparityMap = false;
    this->stereo->flags.show3Dreconstruction = false;
    this->stereo->flags.showTrackingObjectView = false;
    this->stereo->flags.showDiffImage = true;
    this->stereo->flags.showWarningLines = false;
}

void MainWindow::on_btnShowDiffImage_2_clicked(){
    this->stereo->flags.showInputImages = false;
    this->stereo->flags.showDisparityMap = false;
    this->stereo->flags.show3Dreconstruction = false;
    this->stereo->flags.showTrackingObjectView = false;
    this->stereo->flags.showDiffImage = false;
    this->stereo->flags.showWarningLines = true;
}

QImage MainWindow::putImage(const Mat& mat){
    // 8-bits unsigned, NO. OF CHANNELS=1
    if(mat.type()==CV_8UC1)
    {
        // Set the color table (used to translate colour indexes to qRgb values)
        QVector<QRgb> colorTable;
        for (int i=0; i<256; i++)
            colorTable.push_back(qRgb(i,i,i));
        // Copy input Mat
        const uchar *qImageBuffer = (const uchar*)mat.data;
        // Create QImage with same dimensions as input Mat
        QImage img(qImageBuffer, mat.cols, mat.rows, mat.step, QImage::Format_Indexed8);
        img.setColorTable(colorTable);
        return img;
    }
    // 8-bits unsigned, NO. OF CHANNELS=3
    if(mat.type()==CV_8UC3)
    {
        // Copy input Mat
        const uchar *qImageBuffer = (const uchar*)mat.data;
        // Create QImage with same dimensions as input Mat
        QImage img(qImageBuffer, mat.cols, mat.rows, mat.step, QImage::Format_RGB888);
        return img.rgbSwapped();
    }
    else
    {
        qDebug() << "ERROR: Mat could not be converted to QImage.";
        return QImage();
    }
}

void writeMatToFile(cv::Mat& m, const char* filename)
{
    ofstream fout(filename);

    if(!fout)
    {
        cout<<"File Not Opened"<<endl;  return;
    }

    for(int i=0; i<m.rows; i++)
    {
        for(int j=0; j<m.cols; j++)
        {
            fout<<m.at<float>(i,j)<<"\t";
        }
        fout<<endl;
    }

    fout.close();
}

void MainWindow::createTrackbars(){ //Create Window for trackbars
    char TrackbarName[50];

    // Create TrackBars Window
    namedWindow(trackbarWindowName,0);

    // Create memory to store Trackbar name on window
    sprintf( TrackbarName, "preFilterSize");
    sprintf( TrackbarName, "preFilterCap");
    sprintf( TrackbarName, "SADWindowSize");
    sprintf( TrackbarName, "minDisparity");
    sprintf( TrackbarName, "numberOfDisparities");
    sprintf( TrackbarName, "textureThreshold");
    sprintf( TrackbarName, "uniquenessRatio");
    sprintf( TrackbarName, "speckleWindowSize");
    sprintf( TrackbarName, "speckleRange");
    sprintf( TrackbarName, "disp12MaxDiff");

    //Create Trackbars and insert them into window


    createTrackbar( "preFilterSize", trackbarWindowName, &this->stereo->stereocfg.preFilterSize, preFilterSize_MAX, on_trackbar );
    createTrackbar( "preFilterCap", trackbarWindowName, &this->stereo->stereocfg.preFilterCap, preFilterCap_MAX, on_trackbar );
    createTrackbar( "SADWindowSize", trackbarWindowName, &this->stereo->stereocfg.SADWindowSize, SADWindowSize_MAX, on_trackbar );
    createTrackbar( "minDisparity", trackbarWindowName, &this->stereo->stereocfg.minDisparity, minDisparity_MAX, on_trackbar );
    createTrackbar( "numberOfDisparities", trackbarWindowName, &this->stereo->stereocfg.numberOfDisparities, numberOfDisparities_MAX, on_trackbar );
    createTrackbar( "textureThreshold", trackbarWindowName, &this->stereo->stereocfg.textureThreshold, textureThreshold_MAX, on_trackbar );
    createTrackbar( "uniquenessRatio", trackbarWindowName, &this->stereo->stereocfg.uniquenessRatio, uniquenessRatio_MAX, on_trackbar );
    createTrackbar( "speckleWindowSize", trackbarWindowName, &this->stereo->stereocfg.speckleWindowSize, speckleWindowSize_MAX, on_trackbar );
    createTrackbar( "speckleRange", trackbarWindowName, &this->stereo->stereocfg.speckleRange, speckleRange_MAX, on_trackbar );
    createTrackbar( "disp12MaxDiff", trackbarWindowName, &this->stereo->stereocfg.disp12MaxDiff, disp12MaxDiff_MAX, on_trackbar );
}

void on_trackbar(int,void*){}; //This function gets called whenever a trackbar position is changed

void resizeFrames(Mat* frame1,Mat* frame2){
    if(frame1->cols != 0 || !frame2->cols != 0){
#ifdef RESOLUTION_320x240
        resize(*frame1, *frame1, Size(320,240), 0, 0, INTER_CUBIC);
        resize(*frame2, *frame2, Size(320,240), 0, 0, INTER_CUBIC);
#endif

#ifdef RESOLUTION_640x480
        resize(*frame1, *frame1, Size(640,480), 0, 0, INTER_CUBIC);
        resize(*frame2, *frame2, Size(640,480), 0, 0, INTER_CUBIC);
#endif

#ifdef RESOLUTION_1280x720
        resize(*frame1, *frame1, Size(1280,720), 0, 0, INTER_CUBIC);
        resize(*frame2, *frame2, Size(1280,720), 0, 0, INTER_CUBIC);
#endif
    }
}

void change_resolution(VideoCapture* capL,VideoCapture* capR){
#ifdef RESOLUTION_320x240
    capL->set(CV_CAP_PROP_FRAME_WIDTH, 320);
    capL->set(CV_CAP_PROP_FRAME_HEIGHT,240);
    capR->set(CV_CAP_PROP_FRAME_WIDTH, 320);
    capR->set(CV_CAP_PROP_FRAME_HEIGHT,240);
#endif

#ifdef RESOLUTION_640x480
    capL->set(CV_CAP_PROP_FRAME_WIDTH, 640);
    capL->set(CV_CAP_PROP_FRAME_HEIGHT,480);
    capR->set(CV_CAP_PROP_FRAME_WIDTH, 640);
    capR->set(CV_CAP_PROP_FRAME_HEIGHT,480);
#endif

#ifdef RESOLUTION_1280x960
    capL->set(CV_CAP_PROP_FRAME_WIDTH,1280);
    capL->set(CV_CAP_PROP_FRAME_HEIGHT,720);
    capR->set(CV_CAP_PROP_FRAME_WIDTH,1280);
    capR->set(CV_CAP_PROP_FRAME_HEIGHT,720);
#endif

    cout << "Camera 1 Resolution: " << capL->get(CV_CAP_PROP_FRAME_WIDTH) << "x" << capL->get(CV_CAP_PROP_FRAME_HEIGHT) << endl;
    cout << "Camera 2 Resolution: " << capR->get(CV_CAP_PROP_FRAME_WIDTH) << "x" << capR->get(CV_CAP_PROP_FRAME_HEIGHT) << endl;
}

void imageProcessing1(Mat Image, Mat MedianImage, Mat MedianImageBGR){

    // Apply Median Filter
    medianBlur(Image,MedianImage,5);
    applyColorMap(MedianImage,MedianImageBGR, COLORMAP_JET);

    // Output
    imshow("Disparity Map Median Filter 3x3",MedianImage);
    imshow("Disparity Map Median Filter 3x3 - RGB",MedianImageBGR);
}

void contrast_and_brightness(Mat &left,Mat &right,float alpha,float beta){
    //Contrast and Brightness. Do the operation: new_image(i,j) = alpha*image(i,j) + beta
    for( int y = 0; y < left.rows; y++ ){
        for( int x = 0; x < left.cols; x++ ){
            for( int c = 0; c < 3; c++ ){
                left .at<Vec3b>(y,x)[c] = saturate_cast<uchar>( alpha*( left .at<Vec3b>(y,x)[c] ) + beta );
                right.at<Vec3b>(y,x)[c] = saturate_cast<uchar>( alpha*( right.at<Vec3b>(y,x)[c] ) + beta );
            }
        }
    }
}
\end{lstlisting}

\textbf{setstereoparams.cpp}
\begin{lstlisting}
/*
 * setstereoparams.cpp
 *
 *  Created on: Oct 1, 2015
 *      Author: nicolasrosa
 */
#include "setstereoparams.h"
#include "ui_setstereoparams.h"
#include "iostream"
#include "StereoProcessor.h"

#include <QHBoxLayout>
#include <QSlider>
#include <QSpinBox>

//using namespace cv;
using namespace std;

SetStereoParams::SetStereoParams(QWidget *parent, StereoProcessor *stereo) : QDialog(parent), ui(new Ui::SetStereoParams){
    ui->setupUi(this);
    //stereocfg_pointer = &stereocfg;

    this->stereo = stereo;

    connect(ui->preFilterSize_slider, SIGNAL(valueChanged(int)),ui->preFilterSize_spinBox,SLOT(setValue(int)));
    connect(ui->preFilterCap_slider, SIGNAL(valueChanged(int)),ui->preFilterCap_spinBox,SLOT(setValue(int)));
    connect(ui->SADWindowSize_slider, SIGNAL(valueChanged(int)),ui->SADWindowSize_spinBox,SLOT(setValue(int)));
    connect(ui->minDisparity_slider, SIGNAL(valueChanged(int)),ui->minDisparity_spinBox,SLOT(setValue(int)));
    connect(ui->numberOfDisparities_slider, SIGNAL(valueChanged(int)),ui->numberOfDisparities_spinBox,SLOT(setValue(int)));
    connect(ui->textureThreshold_slider, SIGNAL(valueChanged(int)),ui->textureThreshold_spinBox,SLOT(setValue(int)));
    connect(ui->uniquenessRatio_slider, SIGNAL(valueChanged(int)),ui->uniquenessRatio_spinBox,SLOT(setValue(int)));
    connect(ui->speckleWindowSize_slider, SIGNAL(valueChanged(int)),ui->speckleWindowSize_spinBox,SLOT(setValue(int)));
    connect(ui->speckleRange_slider, SIGNAL(valueChanged(int)),ui->speckleRange_spinBox,SLOT(setValue(int)));
    connect(ui->disp12MaxDiff_slider, SIGNAL(valueChanged(int)),ui->disp12MaxDiff_spinBox,SLOT(setValue(int)));
    connect(ui->preFilterSize_spinBox, SIGNAL(valueChanged(int)),ui->preFilterSize_slider,SLOT(setValue(int)));
    connect(ui->preFilterCap_spinBox, SIGNAL(valueChanged(int)),ui->preFilterCap_slider,SLOT(setValue(int)));
    connect(ui->SADWindowSize_spinBox, SIGNAL(valueChanged(int)),ui->SADWindowSize_slider,SLOT(setValue(int)));
    connect(ui->minDisparity_spinBox, SIGNAL(valueChanged(int)),ui->minDisparity_slider,SLOT(setValue(int)));
    connect(ui->numberOfDisparities_spinBox, SIGNAL(valueChanged(int)),ui->numberOfDisparities_slider,SLOT(setValue(int)));
    connect(ui->textureThreshold_spinBox, SIGNAL(valueChanged(int)),ui->textureThreshold_slider,SLOT(setValue(int)));
    connect(ui->uniquenessRatio_spinBox, SIGNAL(valueChanged(int)),ui->uniquenessRatio_slider,SLOT(setValue(int)));
    connect(ui->speckleWindowSize_spinBox, SIGNAL(valueChanged(int)),ui->speckleWindowSize_slider,SLOT(setValue(int)));
    connect(ui->speckleRange_spinBox, SIGNAL(valueChanged(int)),ui->speckleRange_slider,SLOT(setValue(int)));
    connect(ui->disp12MaxDiff_spinBox, SIGNAL(valueChanged(int)),ui->disp12MaxDiff_slider,SLOT(setValue(int)));
}

//Ui::SetStereoParams* SetStereoParams::getUi(){
//    return(this->ui);
//}

void SetStereoParams::loadStereoParamsUi(int preFilterSize,int preFilterCap,int SADWindowSize,int minDisparity,int numberOfDisparities,int textureThreshold,int uniquenessRatio, int speckleWindowSize, int speckleRange,int disp12MaxDiff){
    this->ui->preFilterSize_slider->setValue(preFilterSize);
    this->ui->preFilterSize_spinBox->setValue(preFilterSize);

    this->ui->preFilterCap_slider->setValue(preFilterCap);
    this->ui->preFilterCap_spinBox->setValue(preFilterCap);

    this->ui->SADWindowSize_slider->setValue(SADWindowSize);
    this->ui->SADWindowSize_spinBox->setValue(SADWindowSize);

    this->ui->minDisparity_slider->setValue(minDisparity);
    this->ui->minDisparity_spinBox->setValue(minDisparity);

    this->ui->numberOfDisparities_slider->setValue(numberOfDisparities);
    this->ui->numberOfDisparities_spinBox->setValue(numberOfDisparities);

    this->ui->textureThreshold_slider->setValue(textureThreshold);
    this->ui->textureThreshold_spinBox->setValue(textureThreshold);

    this->ui->uniquenessRatio_slider->setValue(uniquenessRatio);
    this->ui->uniquenessRatio_spinBox->setValue(uniquenessRatio);

    this->ui->speckleWindowSize_slider->setValue(speckleWindowSize);
    this->ui->speckleWindowSize_spinBox->setValue(speckleWindowSize);

    this->ui->speckleRange_slider->setValue(speckleRange);
    this->ui->speckleRange_spinBox->setValue(speckleRange);

    this->ui->disp12MaxDiff_slider->setValue(disp12MaxDiff);
    this->ui->disp12MaxDiff_spinBox->setValue(disp12MaxDiff);
}

//void SetStereoParams::getStereoParamsUi(){

//}

SetStereoParams::~SetStereoParams()
{
    delete ui;
}

/* Sliders */
void SetStereoParams::on_preFilterSize_slider_valueChanged(int value)
{
    cout << "Bar1: " << value << endl;
    updateValues();
}

void SetStereoParams::on_preFilterCap_slider_valueChanged(int value)
{
    cout << "Bar2: " << value << endl;
    updateValues();
}

void SetStereoParams::on_SADWindowSize_slider_valueChanged(int value)
{
    cout << "Bar3: " << value << endl;
    updateValues();
}

void SetStereoParams::on_minDisparity_slider_valueChanged(int value)
{
    cout << "Bar4: " << value << endl;
    updateValues();
}

void SetStereoParams::on_numberOfDisparities_slider_valueChanged(int value)
{
    cout << "Bar5: " << value << endl;
    updateValues();
}

void SetStereoParams::on_textureThreshold_slider_valueChanged(int value)
{
    cout << "Bar6: " << value << endl;
    updateValues();
}

void SetStereoParams::on_uniquenessRatio_slider_valueChanged(int value)
{
    cout << "Bar7: " << value << endl;
    updateValues();
}

void SetStereoParams::on_speckleWindowSize_slider_valueChanged(int value)
{
    cout << "Bar8: " << value << endl;
    updateValues();
}

void SetStereoParams::on_speckleRange_slider_valueChanged(int value)
{
    cout << "Bar9: " << value << endl;
    updateValues();
}

void SetStereoParams::on_disp12MaxDiff_slider_valueChanged(int value)
{
    cout << "Bar10: " << value << endl;
    updateValues();
}

/* SpinBoxes */
void SetStereoParams::on_preFilterSize_spinBox_valueChanged(int value)
{
    cout << "Spin1: " << value << endl;
    updateValues();
}

void SetStereoParams::on_preFilterCap_spinBox_valueChanged(int value)
{
    cout << "Spin2: " << value << endl;
    updateValues();
}

void SetStereoParams::on_SADWindowSize_spinBox_valueChanged(int value)
{
    cout << "Spin3: " << value << endl;
    updateValues();
}

void SetStereoParams::on_minDisparity_spinBox_valueChanged(int value)
{
    cout << "Spin4: " << value << endl;
    updateValues();
}

void SetStereoParams::on_numberOfDisparities_spinBox_valueChanged(int value)
{
    cout << "Spin5: " << value << endl;
    updateValues();
}

void SetStereoParams::on_textureThreshold_spinBox_valueChanged(int value)
{
    cout << "Spin6: " << value << endl;
    updateValues();
}

void SetStereoParams::on_uniquenessRatio_spinBox_valueChanged(int value)
{
    cout << "Spin7: " << value << endl;
    updateValues();
}

void SetStereoParams::on_speckleWindowSize_spinBox_valueChanged(int value)
{
    cout << "Spin8: " << value << endl;
    updateValues();
}

void SetStereoParams::on_speckleRange_spinBox_valueChanged(int value)
{
    cout << "Spin9: " << value << endl;
    updateValues();
}

void SetStereoParams::on_disp12MaxDiff_spinBox_valueChanged(int value)
{
    cout << "Spin10: " << value << endl;
    updateValues();
}

void SetStereoParams::on_buttonBox_accepted(){

}

void SetStereoParams::on_buttonBox_rejected(){

}

void SetStereoParams::updateValues() {

    std::cout << "UPDATE VALUES!\n";
    stereo->setValues(ui->preFilterSize_slider->value(),
                      ui->preFilterCap_slider->value(),
                      ui->SADWindowSize_slider->value(),
                      ui->minDisparity_slider->value(),
                      ui->numberOfDisparities_slider->value(),
                      ui->textureThreshold_slider->value(),
                      ui->uniquenessRatio_slider->value(),
                      ui->speckleWindowSize_slider->value(),
                      ui->speckleRange_slider->value(),
                      ui->disp12MaxDiff_slider->value());

    this->stereo->setStereoParams();
}
\end{lstlisting}

\textbf{StereoCalib.cpp}
\begin{lstlisting}
/*
 * StereoCalib.cpp
 *
 *  Created on: Dec 3, 2015
 *      Author: nicolasrosa
 */

#include "StereoCalib.h"

/* Constructor */
StereoCalib::StereoCalib(){}

void StereoCalib::createKMatrix(){
    this->K=Mat::eye(3,3,CV_64F);
    this->K.at<double>(0,0)=this->focalLength;
    this->K.at<double>(1,1)=this->focalLength;
    //this->K.at<double>(0,2)=(this->imageSize.width-1.0)/2.0;
    //this->K.at<double>(1,2)=(this->imageSize.height-1.0)/2.0;
    this->K.at<double>(0,2)=(0-1.0)/2.0;
    this->K.at<double>(1,2)=(0-1.0)/2.0;
    cout << "K:" << endl << this->K << endl;
}

/*** Read Q Matrix function
  ** Description: Reads the Q Matrix in the *.yml file
  ** Receives:    Matrices Addresses for storage
  ** Returns:     Nothing
  **
  ** Perspective transformation matrix(Q)
  ** [ 1  0    0	   -cx     ]
  ** [ 0  1    0 	   -cy     ]
  ** [ 0  0    0		f      ]
  ** [ 0  0  -1/Tx 	(cx-cx')/Tx]
  ***/
void StereoCalib::readQMatrix(){
    FileStorage fs(this->QmatrixFileName, FileStorage::READ);

    if(this->is640x480){
        if(!fs.isOpened()){
            cerr << "Failed to open Q.yml file" << endl;
            return;
        }

        fs["Q"] >> this->Q;
        // Check
        if(!this->Q.data){
            cerr << "Check Q Matrix Content!" << endl;
            return;
        }

        // Display
        cout << "Q:" << endl << this->Q << endl;

        this->focalLength = this->Q.at<double>(2,3);  cout << "f:" << this->focalLength << endl;
        this->baseline = -1.0/this->Q.at<double>(3,2); cout << "baseline: " << this->baseline << endl;
    }else{
        cerr << "Check Q.yml file!\n" << endl;
        return;
    }
}

void StereoCalib::calculateQMatrix(){
    this->Q = Mat::eye(4,4,CV_64F);
    this->Q.at<double>(0,3)=-this->imageCenter.x;
    this->Q.at<double>(1,3)=-this->imageCenter.y;
    this->Q.at<double>(2,3)=this->focalLength;
    this->Q.at<double>(3,3)=0.0;
    this->Q.at<double>(2,2)=0.0;
    this->Q.at<double>(3,2)=1.0/this->baseline;
    cout << "Q:" << endl << this->Q << endl;
}
\end{lstlisting}

\textbf{StereoConfig.cpp}
\begin{lstlisting}
/*
 * StereoConfig.cpp
 *
 *  Created on: Dec 3, 2015
 *      Author: nicolasrosa
 */

#include "StereoConfig.h"

/* Constructor */
StereoConfig::StereoConfig(){}
\end{lstlisting}

\textbf{StereoCustom.h}
\begin{lstlisting}
/*
 * StereoCustom.h
 *
 *  Created on: Dec 3, 2015
 *      Author: nicolasrosa
 */

#include "StereoCustom.h"
\end{lstlisting}

\textbf{StereoDiff.cpp}
\begin{lstlisting}
/*
 * StereoDiff.cpp
 *
 *  Created on: Dec 3, 2015
 *      Author: nicolasrosa
 */

#include "StereoDiff.h"

/* Constructor */
StereoDiff::StereoDiff(){
    StartDiff=false;
    alpha = 0.5;
    beta = (1.0-alpha);
}

void StereoDiff::createDiffImage(Mat input1, Mat input2){
    absdiff(input1,input2,diffImage);
}

void StereoDiff::createResAND(Mat input1,Mat input2){
    bitwise_and(input1,input2,this->res_AND);
}

void StereoDiff::convertToBGR(){
    cvtColor(res_AND,res_AND_BGR,CV_GRAY2BGR);
}

void StereoDiff::addRedLines(){
    split(res_AND_BGR,res_AND_BGR_channels);

    //Set the Blue and Green Channels to 0
    res_AND_BGR_channels[0] = Mat::zeros(res_AND.rows,res_AND.cols,CV_8UC1);
    res_AND_BGR_channels[1] = Mat::zeros(res_AND.rows,res_AND.cols,CV_8UC1);
    cv::merge(res_AND_BGR_channels,3,res_AND_BGR);

    addWeighted(this->imageL,alpha,res_AND_BGR,beta, 0.0,res_ADD);

    //imshow("Add",res_ADD);
}
\end{lstlisting}

\textbf{StereoDisparityMap.cpp}
\begin{lstlisting}
/*
 * StereoDisparityMap.cpp
 *
 *  Created on: Dec 3, 2015
 *      Author: nicolasrosa
 */

#include "StereoDisparityMap.h"

/* Constructor */
StereoDisparityMap::StereoDisparityMap(){
    // Allocate Memory
    //Mat disp     = Mat(imageR[0].rows, imageR[0].cols, CV_16UC1);
    //Mat disp_8U  = Mat(stereo->imageR[0].rows, stereo->imageR[0].cols, CV_8UC1);
    //Mat disp_BGR = Mat(stereo->imageR[0].rows, stereo->imageR[0].cols, CV_8UC3);
}
\end{lstlisting}

\textbf{StereoFlags.cpp}
\begin{lstlisting}
/*
 * StereoFlags.cpp
 *
 *  Created on: Dec 3, 2015
 *      Author: nicolasrosa
 */

#include "StereoFlags.h"

/* Constructor */
StereoFlags::StereoFlags(){
    showInputImages=true;
    showXYZ=false;
    showStereoParam=false;
    showStereoParamValues=false;
    showFPS=false;
    showDisparityMap=false;
    show3Dreconstruction=false;
    showTrackingObjectView=false;
    showDiffImage=false;
    showWarningLines=false;
}
\end{lstlisting}

\textbf{StereoProcessor.cpp}
\begin{lstlisting}
/*
 * StereoProcessor.cpp
 *
 *  Created on: Oct 20, 2015
 *      Author: nicolasrosa
 */

#include "trackObject.h"

/* Constructor */
#include "StereoProcessor.h"

StereoProcessor::StereoProcessor(int number) {
    inputNum=number;
}

int StereoProcessor::getInputNum(){
    return inputNum;
}

void StereoProcessor::readConfigFile(){
    //FileStorage fs("../reprojectImageTo3D_calibON_bm_GUI/config.yml", FileStorage::READ);
    FileStorage fs("/home/nicolas/repository/StereoVision/Qt Creator/reprojectImageTo3D_calibON_bm_GUI/config.yml", FileStorage::READ);

    if(!fs.isOpened()){
        cerr << "Failed to open config.yml file!" << endl;
        return;
    }
    fs["Intrinsics Path"] >> this->calib.intrinsicsFileName;
    fs["Extrinsics Path"] >> this->calib.extrinsicsFileName;
    fs["Q Matrix Path"]   >> this->calib.QmatrixFileName;
    fs["Stereo Parameters Path"] >> this->calib.StereoParamFileName;

    fs.release();

    cout << "------------------------------Config.yml------------------------------" << endl;
    cout << "Intrinsics Path: "         << this->calib.intrinsicsFileName  << endl;
    cout << "Extrinsics Path: "         << this->calib.extrinsicsFileName  << endl;
    cout << "Q Matrix Path: "           << this->calib.QmatrixFileName     << endl;
    cout << "Stereo Parameters Path:"   << this->calib.StereoParamFileName << endl;
    cout << "Config.yml Read Successfully." << endl << endl ;
    cout << "----------------------------------------------------------------------" << endl;
}

void StereoProcessor::readStereoConfigFile(){
    FileStorage fs(this->calib.StereoParamFileName, FileStorage::READ);
    if(!fs.isOpened()){
        cerr << "Failed to open stereo.yml file!" << endl;
        return;
    }

    fs["preFilterSize"] >> this->stereocfg.preFilterSize;
    fs["preFilterCap"] >> this->stereocfg.preFilterCap;
    fs["SADWindowSize"] >> this->stereocfg.SADWindowSize;
    fs["minDisparity"] >> this->stereocfg.minDisparity;
    fs["numberOfDisparities"] >> this->stereocfg.numberOfDisparities;
    fs["textureThreshold"] >> this->stereocfg.textureThreshold;
    fs["uniquenessRatio"] >> this->stereocfg.uniquenessRatio;
    fs["speckleWindowSize"] >> this->stereocfg.speckleWindowSize;
    fs["speckleRange"] >> this->stereocfg.speckleRange;
    fs["disp12MaxDiff"] >> this->stereocfg.disp12MaxDiff;

    fs.release();

    // Display
    cout << "------------------------------StereoConfig----------------------------" << endl;
    cout << "preFilterSize: "       << this->stereocfg.preFilterSize          << endl;
    cout << "preFilterCap: "        << this->stereocfg.preFilterCap           << endl;
    cout << "SADWindowSize: "       << this->stereocfg.SADWindowSize          << endl;
    cout << "minDisparity: "        << this->stereocfg.minDisparity           << endl;
    cout << "numberOfDisparities: " << this->stereocfg.numberOfDisparities    << endl;
    cout << "textureThreshold: "    << this->stereocfg.textureThreshold       << endl;
    cout << "uniquenessRatio: "     << this->stereocfg.uniquenessRatio        << endl;
    cout << "speckleWindowSize: "   << this->stereocfg.speckleWindowSize      << endl;
    cout << "speckleRange: "        << this->stereocfg.speckleRange           << endl;
    cout << "disp12MaxDiff: "       << this->stereocfg.disp12MaxDiff          << endl;
    cout << "stereo.yml Read Successfully."  << endl << endl;
    cout << "----------------------------------------------------------------------" << endl << endl;
}

/*** Stereo Initialization function
  ** Description: Executes the PreSetup of parameters of the StereoBM object
  ** @param StereoBM bm: Correspondence Object
  ** Returns:     Nothing
  ***/
void StereoProcessor::stereoInit(){
    this->bm->setPreFilterCap(31);
    this->bm->setBlockSize(25 > 0 ? 25 : 9);
    this->bm->setMinDisparity(0);
    this->bm->setNumDisparities(3);
    this->bm->setTextureThreshold(10);
    this->bm->setUniquenessRatio(15);
    this->bm->setSpeckleWindowSize(100);
    this->bm->setSpeckleRange(32);
    this->bm->setDisp12MaxDiff(1);
}

/*** Stereo Calibration function
  ** Description: Reads the Calibrations in *.yml files
  ** Receives:    Matrices Addresses for storage
  ** @param Mat M1,M2: Intrinsic Matrices from camera 1 and 2
  ** @param Mat D1,D2: Distortion Coefficients from camera 1 and 2
  ** @param Mat R: Rotation Matrix
  ** @param Mat t: Translation Vector
  ** Returns:     Nothing
  ***/
void StereoProcessor::stereoCalib(){
    FileStorage fs(this->calib.intrinsicsFileName, FileStorage::READ);
    if(!fs.isOpened()){
        cerr << "Failed to open intrinsics.yml file!" << endl;
        return;
    }

    fs["M1"] >> this->calib.M1;
    fs["D1"] >> this->calib.D1;
    fs["M2"] >> this->calib.M2;
    fs["D2"] >> this->calib.D2;

    fs.release();

    float scale = 1.f;
    this->calib.M1 *= scale;
    this->calib.M2 *= scale;

    fs.open(this->calib.extrinsicsFileName, FileStorage::READ);
    if(!fs.isOpened()){
        cerr << "Failed to open extrinsics.yml file!" << endl;
        return;
    }

    fs["R"] >> this->calib.R;
    fs["T"] >> this->calib.T;

    fs.release();

    // Check
    if(!this->calib.M1.data || !this->calib.D1.data || !this->calib.M2.data || !this->calib.D2.data || !this->calib.R.data || !this->calib.T.data){
        cerr << "Check instrinsics and extrinsics Matrixes content!" << endl;
        return;
    }

    // Display
    cout << "------------------------------Intrinsics------------------------------" << endl;
    cout << "M1: " << endl << this->calib.M1 << endl;
    cout << "D1: " << endl << this->calib.D1 << endl;
    cout << "M2: " << endl << this->calib.M2 << endl;
    cout << "D2: " << endl << this->calib.D2 << endl << endl;
    cout << "intrinsics.yml Read Successfully."  << endl << endl;

    cout << "------------------------------Extrinsics------------------------------" << endl;
    cout << "R: " << endl << this->calib.R << endl;
    cout << "T: " << endl << this->calib.T << endl << endl;
    cout << "extrinsics.yml Read Successfully."  << endl;
    cout << "----------------------------------------------------------------------" << endl << endl;
}

/*** Stereo Parameters Configuration function
  ** Description: Executes the setup of parameters of the StereoBM object by changing the trackbars
  ** @param rect roi1: Region of Interest 1
  ** @param rect roi2: Region of Interest 2
  ** @param StereoBM bm: Correspondence Object
  ** @param int numRows: Number of Rows of the input Images
  ** @param bool showStereoBMparams
  ** Returns:     Nothing
  ***/
void StereoProcessor::setStereoParams(){
    int trackbarsAux[10];

    trackbarsAux[0] = this->stereocfg.preFilterSize*2.5+5;
    trackbarsAux[1] = this->stereocfg.preFilterCap*0.625+1;
    trackbarsAux[2] = this->stereocfg.SADWindowSize*2.5+5;
    trackbarsAux[3] = this->stereocfg.minDisparity*2.0-100;
    trackbarsAux[4] = this->stereocfg.numberOfDisparities*16;
    trackbarsAux[5] = this->stereocfg.textureThreshold*320;
    trackbarsAux[6] = this->stereocfg.uniquenessRatio*2.555;
    trackbarsAux[7] = this->stereocfg.speckleWindowSize*1.0;
    trackbarsAux[8] = this->stereocfg.speckleRange*1.0;
    trackbarsAux[9] = this->stereocfg.disp12MaxDiff*1.0;

    this->bm->setROI1(this->calib.roi1);
    this->bm->setROI1(this->calib.roi2);

    this->numRows = imageL[0].rows;

    if(trackbarsAux[0]%2==1 && trackbarsAux[0]>=5 && trackbarsAux[0]<=255){
        //bm.state->preFilterSize = trackbarsAux[0];
        bm->setPreFilterSize(trackbarsAux[0]);
    }

    if(trackbarsAux[1]>=1 && trackbarsAux[1]<=63){
        //bm.state->preFilterCap = trackbarsAux[1];
        bm->setPreFilterCap(trackbarsAux[1]);
    }

    if(trackbarsAux[2]%2==1 && trackbarsAux[2]>=5  && trackbarsAux[2]<=255 && trackbarsAux[2]<=numRows){
        //bm.state->SADWindowSize = trackbarsAux[2];
        bm->setBlockSize(trackbarsAux[2]);
    }

    if(trackbarsAux[3]>=-100 && trackbarsAux[3]<=100){
        //bm.state->minDisparity = trackbarsAux[3];
        bm->setMinDisparity(trackbarsAux[3]);
    }

    if(trackbarsAux[4]%16==0 && trackbarsAux[4]>=16 && trackbarsAux[4]<=256){
        //bm.state->numberOfDisparities = trackbarsAux[4];
        bm->setNumDisparities(trackbarsAux[4]);
    }

    if(trackbarsAux[5]>=0 && trackbarsAux[5]<=32000){
        //bm.state->textureThreshold = trackbarsAux[5];
        bm->setTextureThreshold(trackbarsAux[5]);
    }

    if(trackbarsAux[6]>=0 && trackbarsAux[6]<=255){
        //bm.state->uniquenessRatio = trackbarsAux[6];
        bm->setUniquenessRatio(trackbarsAux[6]);
    }

    if(trackbarsAux[7]>=0 && trackbarsAux[7]<=100){
        //bm.state->speckleWindowSize = trackbarsAux[7];
        bm->setSpeckleWindowSize(trackbarsAux[7]);
    }

    if(trackbarsAux[8]>=0 && trackbarsAux[8]<=100){
        //bm.state->speckleRange = trackbarsAux[8];
        bm->setSpeckleRange(trackbarsAux[8]);
    }

    if(trackbarsAux[9]>=0 && trackbarsAux[9]<=100){
        //bm.state->disp12MaxDiff = trackbarsAux[9];
        bm->setDisp12MaxDiff(trackbarsAux[9]);
    }

    if(showStereoParamsValues){
        cout << getTrackbarPos("preFilterSize",trackbarWindowName)			<< "\t" << trackbarsAux[0] << endl;
        cout << getTrackbarPos("preFilterCap",trackbarWindowName)			<< "\t" << trackbarsAux[1] << endl;
        cout << getTrackbarPos("SADWindowSize",trackbarWindowName)			<< "\t" << trackbarsAux[2] << endl;
        cout << getTrackbarPos("minDisparity",trackbarWindowName)			<< "\t" << trackbarsAux[3] << endl;
        cout << getTrackbarPos("numberOfDisparities",trackbarWindowName)	<< "\t" << trackbarsAux[4] << endl;
        cout << getTrackbarPos("textureThreshold",trackbarWindowName)		<< "\t" << trackbarsAux[5] << endl;
        cout << getTrackbarPos("uniquenessRatio",trackbarWindowName)		<< "\t" << trackbarsAux[6] << endl;
        cout << getTrackbarPos("speckleWindowSize",trackbarWindowName)		<< "\t" << trackbarsAux[7] << endl;
        cout << getTrackbarPos("speckleRange",trackbarWindowName)			<< "\t" << trackbarsAux[8] << endl;
        cout << getTrackbarPos("disp12MaxDiff",trackbarWindowName)			<< "\t" << trackbarsAux[9] << endl;
    }
}

void StereoProcessor::imageProcessing(Mat src, Mat imgE, Mat imgED,Mat cameraFeedL,bool isTrackingObjects){
    Mat erosionElement = getStructuringElement( MORPH_RECT,Size( 2*EROSION_SIZE + 1, 2*EROSION_SIZE+1 ),Point( EROSION_SIZE, EROSION_SIZE ) );
    Mat dilationElement = getStructuringElement( MORPH_RECT,Size( 2*DILATION_SIZE + 1, 2*DILATION_SIZE+1 ),Point( DILATION_SIZE, DILATION_SIZE ) );
    Mat imgEBGR,imgEDBGR;
    Mat imgEDMedian,imgEDMedianBGR;
    int x,y;

    //Mat imgThreshold;
    static Mat lastimgThreshold;
    int nPixels,nTotal;		  	//static int lastThresholdSum=0;

    // Near Object Detection

    //Prefiltering
    // Apply Erosion and Dilation to take out spurious noise
    erode(src,imgE,erosionElement);
    dilate(imgE,imgED,erosionElement);

    applyColorMap(imgE,imgEBGR, COLORMAP_JET);
    applyColorMap(imgED,imgEDBGR, COLORMAP_JET);

    // Apply Median Filter
    //GaussianBlur(imgED,imgEDMedian,Size(3,3),0,0);
    medianBlur(imgED,imgEDMedian,5);
    applyColorMap(imgEDMedian,imgEDMedianBGR, COLORMAP_JET);

    // Thresholding
    //adaptiveThreshold(imgEDMedian,imgThreshold,255,ADAPTIVE_THRESH_MEAN_C,THRESH_BINARY,11,-1);
    //adaptiveThreshold(imgEDMedian,imgThreshold,255,ADAPTIVE_THRESH_GAUSSIAN_C,THRESH_BINARY,11,0);
    threshold(imgEDMedian, imgThreshold, THRESH_VALUE, 255,THRESH_BINARY);
    erode(imgThreshold,imgThreshold,erosionElement);
    dilate(imgThreshold,imgThreshold,dilationElement);

    // Solving Lighting Noise Problem
    nPixels = sum(imgThreshold)[0]/255;
    nTotal = imgThreshold.total();

    //	cout << "Number of Pixels:" << nPixels << endl;
    //	cout << "Ratio is: " << ((float)nPixels)/nTotal << endl << endl;

    if((((float)nPixels)/nTotal)>0.5){
        //		sleep(1);
        //		cout << "Lighting Noise!!!" << endl;
        //		cout << "Number of Pixels:" << nPixels << endl;
        //		cout << "Ratio is: " << ((float)nPixels)/nTotal << endl << endl;

        // Invalidates the last frame
        imgThreshold = lastimgThreshold;
    }else{
        // Saves the last valid frame
        lastimgThreshold=imgThreshold;
        //lastThresholdSum = CurrentThresholdSum;
    }

    // Output
    //imshow("Eroded Image",imgE);
    //imshow("Eroded Image BGR",imgEBGR);
    //imshow("Eroded+Dilated Image",imgED);
    //imshow("Eroded+Dilated Image BGR",imgEDBGR);
    //imshow("Eroded+Dilated+Median Image",imgEDMedian);
    //imshow("Eroded+Dilated+Median Image BGR",imgEDMedianBGR);

    //imshow("Thresholded Image",imgThreshold);

    // Tracking Object
    if(isTrackingObjects){
        cameraFeedL.copyTo(trackingView);
        trackFilteredObject(x,y,imgThreshold,trackingView);
        //imshow("Tracking Object",trackingView);
    }
}

//Saving Previous Frame
void StereoProcessor::saveLastFrames(){
    imageL[0].copyTo(imageL[1]);
    imageR[0].copyTo(imageR[1]);
    imageL_grey[0].copyTo(imageL_grey[1]);
    imageR_grey[0].copyTo(imageR_grey[1]);
}

void StereoProcessor::setValues(int preFilterSize, int preFilterCap, int sadWindowSize, int minDisparity, int numOfDisparities, int textureThreshold, int uniquenessRatio, int speckleWindowSize, int speckleWindowRange, int disp12MaxDiff) {
    stereocfg.preFilterSize = preFilterSize;
    stereocfg.preFilterCap = preFilterCap;
    stereocfg.SADWindowSize = sadWindowSize;
    stereocfg.minDisparity = minDisparity;
    stereocfg.numberOfDisparities = numOfDisparities;
    stereocfg.textureThreshold = textureThreshold;
    stereocfg.uniquenessRatio = uniquenessRatio;
    stereocfg.speckleRange = speckleWindowRange;
    stereocfg.speckleWindowSize = speckleWindowSize;
    stereocfg.disp12MaxDiff = disp12MaxDiff;

    std::cout << "SET VALUES!\n";
}
\end{lstlisting}

%------------------------------------- Ap�ndice 2 ---------------------------------------------------
\chapter{Ap�ndice 2}
\label{Apendice2}

Texto do Ap�ndice 2.



