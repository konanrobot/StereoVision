\chapter{Introdução}
\label{Introducao}

A pesquisa em veículos aéreos não tripulados (VANTs) vêm se tornando um assunto recorrente no âmbito científico. A real motivação para seu desenvolvimento levanta diversas questões éticas e legais, visto que foram inicialmente motivados para fins militares. Por outro lado, esse tipo de plataforma também possui aplicações tais como: cultivo e pulverização de culturas, produção cinematográfica, operações de busca e salvamento, inspeção de linhas elétricas de alta tensão, entrega de mercadorias e encomendas.

A navegação de um micro veículo aéreo em espaço confinado é um desafio significante. Atualmente, a navegação autônoma enfrenta o problema de localização e mapeamento simultâneo, mais conhecido como SLAM (\textit{Simultaneous Localization and Mapping}) \cite{Dissanayake2001}. O SLAM apresenta quatro etapas: Mapeamento, Percepção, Localização e Modelagem. A complexidade deste problema encontra-se no fato de que o veículo necessita navegar em um espaço desconhecido, extrair características importantes do ambiente, construir um mapa com os dados obtidos e simultaneamente localizar-se dentro deste. O sensoriamento pode ser realizado tanto por visão computacional utilizando câmeras ou por sensores ópticos, como, por exemplo, o LIDAR (Light Detection And Ranging) \cite{Barry2015}. 

O processo de desenvolvimento do veículo consiste em quatro passos: estrutura, circuito, controle e navegação. Os dois primeiros itens compõem o hardware, o qual estabelece as conexões físicas necessárias para integrar os sistemas de alimentação, comunicação e controle. A parte de software engloba o desenvolvimento de algoritmos visando o controle e navegação, mais especificamente o desenvolvimento do código para visão estéreo (\textit{Stereo Vision}), planejamento de caminho (\textit{Path Planning}) e arquitetura de máquina de estados (\textit{Decision Making}) \cite{Lemaire2007}.

Um sistema autônomo também implica que o processamento de navegação, detecção de obstáculos, tomada de decisão, sejam embarcados, isto é, todo processamento necessário deve ser realizado \textit{online}. 

Este trabalho concentra-se no estudo e na implementação do algoritmo de detecção de obstáculos utilizando visão estéreo, tomando as ações cabíveis para que seja embarcado. Deste modo, as plataformas de desenvolvimento BeagleBone Black e NVIDIA Jetson TK1 serão analisadas e suas performances avaliadas ao executar o algoritmo desenvolvido \cite{Shah2014}. 


%-----------------------------------------------------------------------------------------------------------------------------------------------------------------------------------------------
\section{Objetivos}

\begin{enumerate}
	\item Estudo e aplicação de técnicas de visão computacional para visão estéreo.
	\item Desenvolvimento de uma interface de apoio para o monitoramento de um veículo autônomo.
	\item Implementação de algoritmos de visão estéreo para Linux embarcado para aplicação em Quadricópteros.
	\item Comparativo de desempenho do algoritmo implementado em diferentes plataformas.
	\item Estudo e aplicação de métodos para a aceleração em hardware dos algoritmos implementados.
%	\item Utilização da Plataforma ROS (Robot Operating System)	%
\end{enumerate}


%-----------------------------------------------------------------------------------------------------------------------------------------------------------------------------------------------
\section{Justificativa}
% Dica: Explanação sobre porque o trabalho se justifica e quais os pontos de relevância do mesmo.

Há pouco mais de trinta anos, o VANT BQM-1BR realizava seu primeiro voo em espaço aéreo brasileiro. Deste então, mesmo a após a recente popularização dos \textit{Drones}, a legislação com relação à esses veículos ainda é obscura. José Luiz Boa Nova Filho, gerente-adjunto do projeto VANT da Polícia Federal, apresenta o histórico e introduz a atual conjuntura na qual se encontra o processo legislativo \cite{Filho2014}. No dia 19/11/2015, o Departamento de Controle do Espaço Aéreo da Aeronáutica (DECEA) e a Agência Nacional de Aviação Civil (ANAC) publicaram a nova regulamentação para a utilização dos \textit{Remotely-Piloted Aircraft} (RPA), termo adotado para se referir aos VANTs, traduzido como "Aeronave Remotamente Pilotada", substituindo assim a Circular de Informações Aeronáuticas AIC N 21/10. A regulação segue o modelo proposto pela Organização de Aviação Civil Internacional (OACI), a qual preza integralmente pela priorização da segurança, tanto da aeronave quanto dos civis e propriedades \cite{DECEA2015}. 

A \textit{Federal Aviation Administration} (FAA) apresenta as mesmas dificuldades para a integração deste dispositivos em seu espaço aéreo, visto que este é o mais complexo e movimentado do mundo. Deste modo, mesmo sem uma legislação madura, rígidas restrições são impostas para voos em ambientes abertos. Entretanto, assim como as agências brasileiras, a FAA ainda não apresentou uma regulamentação clara envolvendo veículos totalmente autônomos. 

Conclui-se que, mesmo sem um posicionamento concreto das agências reguladoras, a segurança destas aeronaves deve ser priorizada, assim permitindo a utilização e ampliação dessa nova tecnologia. Deste modo, o aprimoramento do sensores para a percepção do ambiente ao redor destes equipamentos torna-se um passo crucial, justificando a execução deste trabalho, o qual estuda a utilização de visão estéreo para a detecção de obstáculos.


%-----------------------------------------------------------------------------------------------------------------------------------------------------------------------------------------------
\section{Motivação}
% Dica: Um pequeno texto sobre o que motivou o desenvolvimento do Trabalho.

A proposta deste trabalho de conclusão de curso é auxiliar o primeiro passo de mapeamento de ambientes através de visão estéreo. Também é motivado pela tentativa de reproduzir-se o desafio proposto pela \textit{Autonomous Aerial Vehicle Competition} (AAVC)\cite{AAVC}, competição organizada pelo Laboratório de Pesquisas da Forca Aérea Americana (AFRL) e sediada em Dayton-OH. Esta competição incentiva o estudo de veículos aéreos autônomos, convidando diversas universidades a compartilhar seus avanços nesta área de pesquisa. O competidor é motivado a adaptar um modelo de quadricóptero $3DRobotics^{\textcopyright}$, assim este veículo precisa cumprir um certo percurso com caixas como obstáculos, detectar e reportar à estação base a posição de um objeto.

A segunda motivação para o desenvolvimento deste trabalho é o crescente número de aplicações de visão estéreo em plataformas robóticas. Atualmente, existem diversas pesquisas voltadas ao desenvolvimento de veículos autônomos para navegação terrestre, aérea e subaquática. A seção \ref{aplicacoes_robotica} traz mais detalhes.


%-----------------------------------------------------------------------------------------------------------------------------------------------------------------------------------------------
\section{Organização do trabalho}
% Dica: Apresente o que tem em cada capítulo.

Esta monografia encontra-se estruturada em 5 capítulos: Introdução, Fundamentos Teóricos, Materiais e Métodos, Resultados e Conclusão. O primeiro capítulo sintetiza o trabalho desenvolvido e apresenta ao leitor suas reais pretensões. O capítulo 2 tem como conteúdo os principais conceitos teóricos para o seu entendimento, onde todo equacionamento e trabalhos similares são apresentados. O terceiro capítulo descreve todos os elementos necessários e técnicas utilizadas para sua realização. O quarto capítulo apresenta os resultados obtidos, onde são interpretados no último capítulo, o de Conclusão.