%-----------------------------------------------------------------------------------------------------------------------------------------------------------------------------------------------
% Conclusões: "fecha" com os objetivos? (respondem aos objetivos?) - aqui é que "se vende o peixe" - elas é que valorizam (ou não) o trabalho realizado.
% Normalmente é uma parte do trabalho "um pouco desprezada", pois o autor já está "cansado....". Mas é aqui que realmente se mede se o trabalho tem ou não valor.
% Contém o item Trabalhos futuros, que é uma orientação sobre as possibilidades de continuação do desenvolvimento do trabalho.
\chapter{Conclusão}
\label{Conclusao}

% TODO: Reescrever para a conclusão Final
Em suma, o trabalho cumpriu todos os objetivos apresentados na seção \ref{objetivos}. 

Com relação à visão estéreo, foi realizado o estudo e a aplicação das técnicas de visão computacional para o problema proposto, mais precisamente estudou-se os processos de calibração, retificação, filtragem e os métodos estéreo para a obtenção de um mapa de disparidades denso. 

Com relação ao monitoramento de um veículo autônomo, foi desenvolvida uma plataforma de apoio que facilita o processo de calibração dos métodos estéreo para as plataformas embarcadas, visto que são destinadas à uma aplicação que não necessita de interface gráfica. Além disso, ela possibilita o monitoramento via estação-base, caso deseja-se que o processamento das imagens não seja realizado \textit{on-board}. Os métodos foram implementados e testados em cenários capturados pela câmera fixada no quadricóptero. Entretanto, a extensão desse trabalho não resume-se a veículos aéreos, as plataformas utilizadas podem muito bem serem utilizadas em veículos terrestres e até mesmo em veículos aquáticos, evidentemente com a devida vedação dos equipamentos. Outro importante objetivo era realização comparação de desempenho dos métodos nas plataformas disponíveis, BBB e Jetson TK1. No caso da BBB, mesmo a rotina desenvolvida mostrando-se totalmente funcional, isto é, os algoritmos consegue detectar e segmentar relativamente bem os obstáculos dos cenários propostos, a taxa de processamento tornou-se uma preocupação. O desempenho foi extremamente baixo, aproximadamente um quadro por segundo, o que claramente inviabiliza sua utilização em uma aplicação real. A alteração de plataforma mostrava-se ser uma solução para o aumento da taxa de atualização, visto que a Jetson TK1 é uma plataforma mais recente e mais poderosa. O desempenho desenvolvido pela Jetson TK1 era de aproximadamente seis FPS (Método BM), performance ainda baixa para um sistemas que deve apresentar resposta em tempo real.

Diante deste impasse, duas providências foram cruciais para a continuidade do trabalho. A primeira foi a otimização das rotinas utilizadas, sendo essa uma tentativa para a melhora do desempenho geral dos métodos. A segunda foi a execução de uma ampla revisão bibliográfica, incluindo principalmente trabalhos que apresentassem comparativos de desempenho entre plataformas e que tivessem realizados algum tipo de aceleração por \textit{hardware}, sejam eles envolvendo paralelização de processos, implementação em FPGA ou utilizando a plataforma de computação paralela CUDA. Ao fim, a ultima alternativa foi tomada visto que a plataforma Jetson TK1 apresenta suporte à tecnologia CUDA.

Com relação à utilização de uma plataforma embarcada em quadricóptero, a plataforma Jetson TK1 juntamente com o método BMGPU foi a melhor alternativa para a implementação em um sistema real, mesmo apresentando uma taxa de performance reduzida, era a melhor solução existente durante o período de desenvolvimento do trabalho. Vale ressaltar que as rotinas implementadas utilizaram os métodos estéreo presentes no OpenCV, isto é, não desenvolveu-se rotinas em programação CUDA, diretivas para a manipulação de dados em GPU. Deste modo, existe a possibilidade de que essas bibliotecas sejam otimizadas em novas versões do OpenCV ou que a NVIDIA atualize a versão do CUDA \textit{Toolkit}.

Com relação à atividades futuras, a NVIDIA lançou recentemente uma nova plataforma, Jetson TX1, a qual apresenta um aumento expressivo de núcleos CUDA, aproximadamente 30\% em relação à versão anterior \cite{JetsonTX1}. A implementação do trabalho desenvolvido nesta nova plataforma é bastante válida, dado a possibilidade de haver um aumento na taxa de processamento. O estudo de novas alternativas para a 



% TODO: Atividades futuras, falar sobre SLAM, falar que atualmente sistema embarcado é a jetson é a melhor alternativa falar pra testar com a nova Jetson TX1, estudar novas alternativas de aceleração em hardware como por exemplo FPGA/ARM, falar que talvez o problema de processamento possa ser resolvido se utilizamos uma camera stereo que ja realize disponibilize o mapa de disparidades e a plataforma fique sómente pra realizar um pós processamento do mapa obtido. falar o quao inviavel eh utilizar a BBB pra aplicação. Falar sobre implementar as diretivas de GPU para um método que apresente maior confiança (pesquisar algo do tipo "SGBM_GPU")
%http://www.geoengine.uni-stuttgart.de/forum/master/hailong_fu/Dense_matching_GPU_implementation_short_Hailong_Fu.pdf
%http://ieeexplore.ieee.org/stamp/stamp.jsp?tp=&arnumber=5606438)

% TODO: Soluções Viaveis - Conclusoes mais importantes do trabalho
% 1 - Por hora, a Jetson é a melhor alternativa pra a implementação de Visao Estéreo, visto que ela permite aceleração em hardware(GPU). Recentemente lancou uma nova versão Jetson TX1, seria interessante estudar o desempenho dela neste contexto.
% 2 - Seria interessante estudar outras artiteturas de hardware como FPGA, e desenvolver um hardware totalmente especializado para esta aplicação. Acredita-se que o desemepnho seja melhor, pois a Jetson mesmo tendo a habilidade de aceleração em hardware, ainda é uma plataforma de desenvolvimento e não de aplicaçoes especializadas.
% 3 - Outra Solução para o problema seria a utilziação de CÂmeras que já forcenem o mapa de disparidades como a camera BumbleBee(Checar se ela realmente tem o mapa de disparidades como saida) e a ZED. Entretanto, deve se preocupar com o peso dos equipamentos. 
% Checar video: https://www.youtube.com/watch?v=n8YsrmxJgBQ


% 4 - Desenvolvimento de Algoritmos que melhorem a confiança do métodos estéreo, por exemplo implementação do metodo SGM em CUDA como mostrado no link:
% http://on-demand.gputechconf.com/gtc/2015/posters/GTC_2015_Computer_Vision___Machine_Vision_21_P5332_WEB.pdf

% 5 - Explicar pq da pequena diferenca de desempenho do Metodo BM e o Metodo BM_GPU na Jetson TK1


%\begin{enumerate}
%	\item Estudo e aplicação de técnicas de visão computacional para visão estéreo.
%	\item Desenvolvimento de uma interface de apoio para o monitoramento de um veículo autônomo.
%	\item Implementação de algoritmos de visão estéreo para Linux embarcado para aplicação em Quadricópteros.
%	\item Comparativo de desempenho do algoritmo implementado em diferentes plataformas.
%	\item Estudo e aplicação de métodos para a aceleração em hardware dos algoritmos implementados.
%	\item Utilização da Plataforma ROS (Robot Operating System)	%
%\end{enumerate}


Atividades Futuras:

\begin{itemize}
	\item Otimizar Código
	\item Alterar Interface gráfica para suportar o Método SGBM
	\item Implementar Segmentação por Distância/Limiarização do mapa de disparidades utilizando Método de Otsu
	\item Aprimorar a suavização da Nuvem de Pontos Gerada
	\item Analisar Métodos de Abertura e Fechamento para aprimoramento na Segmentação
	\item Implementar funções para medir a Distância de objetos próximos
	\item Reavaliar a Calibração das Câmeras	
	\item Realizar ensaio de Ground Truth para validação da Distância Detectada
	\item Implementar os algoritmos de Visão na BeagleBone e Jetson TK1
	\item Implementar os algoritmos de Visão na Jetson TK1
	\item Comparação de Desempenho: Desktop x BBB x Jetson TK1
\end{itemize}



